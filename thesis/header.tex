\usepackage[utf8]{inputenc}
\usepackage{indentfirst}
\usepackage{enumitem}
\usepackage{datetime2}
\usepackage{acronym}
% options (https://tex.stackexchange.com/questions/25520/how-can-i-use-the-latex-acronym-package-and-optionally-create-an-acronym-list-i):
% printonlyused: Only list used acronyms
% withpage: In printonlyused-mode show the page number where each acronym was first used.
% nolist: The option nolist stands for “don’t write the list of acronyms”.
% dua: The option dua stands for “don’t use acronyms”. It leads to a redefinition of \ac and \acp, making the full name appear all the time and suppressing all acronyms but the explicity requested by \acf or \acfp.

\makeatletter
\AtBeginDocument{%
  \renewcommand*{\AC@hyperlink}[2]{%
    \begingroup
      \hypersetup{hidelinks}%
      \hyperlink{#1}{#2}%
    \endgroup
  }%
}
\makeatother

\usepackage{algorithm}
\usepackage{algpseudocode}

\usepackage{textcomp}
\usepackage[T1]{fontenc}
\usepackage{multirow,bigdelim}
\usepackage{float}
\usepackage[caption = false]{subfig}
\usepackage{longtable}
\usepackage{listings}
\usepackage{mathtools}
\DeclareMathOperator{\tr}{Tr}
\usepackage{commath}
\usepackage{slashed}
\usepackage{bbold}
\usepackage{xcolor}
\usepackage{physics}
\newcommand{\lambdabar}{{\mkern0.75mu\mathchar '26\mkern -9.75mu\lambda}}
\usepackage[right=4cm,left=2cm,top=3cm,bottom=3.0cm, marginparwidth=2.7cm, marginparsep=3mm]{geometry}
\usepackage{mdframed}
\usepackage[version=4]{mhchem}
\usepackage{tikz}
\usetikzlibrary{shapes,arrows,shadows}
\newcommand{\mx}[1]{\mathbf{\bm{#1}}} % Matrix command
\newcommand{\vc}[1]{\mathbf{\bm{#1}}} % Vector command

% \usetikzlibrary{shapes,arrows,positioning,automata,backgrounds,calc,er,patterns}
% \usepackage{tikz-feynman}

% \tikzfeynmanset{
%   extra large/.style={
%     /tikz/node distance=3cm,
%     /graph drawing/node distance=4cm,
%     /graph drawing/level distance=3cm,
%     /graph drawing/sibling distance=5cm,
%     /tikz/graphs/edges={thick},
%     /tikzfeynman/every dot@@/.append style={/tikz/minimum size=4mm},
%     /tikzfeynman/every crossed dot@@/.append style={/tikz/minimum size=8mm},
%     /tikzfeynman/every blob@@/.append style={/tikz/minimum size=2cm},
%     /tikzfeynman/arrow size=2pt,
%     /tikzfeynman/insertion/size=8pt,
%   },
% }

% \usepackage{simplewick}
% \usepackage{simpler-wick}
% \usetikzlibrary{calc}

%\usepackage{tikz-cd}
\usepackage{amsmath}

% \newsavebox{\accentbox}
% \newcommand{\compositeaccents}[2]{%
%   \sbox\accentbox{$#2$}#1{\usebox\accentbox}}
% \usepackage[single]{accents}

\usepackage{amsfonts}
\usepackage{amssymb}
\usepackage{amsthm}
\numberwithin{equation}{section}
\usepackage{graphicx}

\usepackage[colorinlistoftodos]{todonotes}
\PassOptionsToPackage{hyphens}{url}
\usepackage[colorlinks=true, allcolors=blue]{hyperref}
\hypersetup{breaklinks=true}

% \urlstyle{same}
\usepackage{siunitx}
\sisetup{separate-uncertainty=true}
\DeclareSIUnit\erg{erg}
\DeclareSIUnit\parsec{pc}
\DeclareSIUnit\littleh{\textit{h}}
\usepackage{cancel}
\usepackage{mathrsfs}
\usepackage{tensor}
\usepackage{marginnote}
\renewcommand*{\marginnotevadjust}{-0.3cm}
\renewcommand*{\marginfont}{\scriptsize}
% \usepackage{fancybox}

\usepackage{footnotebackref}

\usepackage[sc]{mathpazo}
\linespread{1.05}         % Palladio needs more leading (space between lines)
\usepackage[T1]{fontenc}

\newcommand{\diag}[1]{\text{diag}\qty(#1)}
\newcommand{\const}{\text{const}}
\newcommand{\sign}{\text{sign}}
\renewcommand{\H}{\mathcal{H}}
\renewcommand{\dim}{\text{dim}}
\newcommand{\supp}[1]{\text{supp} \qty(#1)}

% \usepackage{nicefrac}
% \usepackage{ifthen}
% \let\oldfrac\frac
% \renewcommand{\frac}[3][d]{\ifthenelse{\equal{#1}{d}}{\oldfrac{#2}{#3}}{\nicefrac{#2}{#3}}}

\renewcommand{\var}[1]{\text{var} \qty(#1)}
\newcommand{\defeq}{\ensuremath{\stackrel{\text{def}}{=}}}

\newcommand\mybox[1]{%
  \fbox{\begin{minipage}{0.9\textwidth}#1\end{minipage}}}

%Spiegazioni/verifiche
\newenvironment{greenbox}{\begin{mdframed}[hidealllines=true,backgroundcolor=green!20,innerleftmargin=3pt,innerrightmargin=3pt]}{\end{mdframed}}

%Approfondimenti
\newenvironment{bluebox}{\begin{mdframed}[hidealllines=true,backgroundcolor=blue!10,innerleftmargin=3pt,innerrightmargin=3pt]}{\end{mdframed}}

\newtheorem{claim}{Claim}[section]
\newtheorem{theorem}{Theorem}[section]
\newtheorem{definition}{Definition}[section]
\newtheorem{proposition}{Proposition}[section]

\usepackage{circledsteps}

\newcommand{\hlc}[2]{%
  \colorbox{#1!50}{$\displaystyle#2$}}

\definecolor{codegreen}{rgb}{0,0.6,0}
\definecolor{codegray}{rgb}{0.5,0.5,0.5}
\definecolor{codepurple}{rgb}{0.58,0,0.82}
\definecolor{backcolour}{rgb}{0.95,0.95,0.92}

\lstdefinestyle{mystyle}{
  backgroundcolor=\color{backcolour},
  commentstyle=\color{codegreen},
  keywordstyle=\color{magenta},
  numberstyle=\tiny\color{codegray},
  stringstyle=\color{codepurple},
  basicstyle=\ttfamily\footnotesize,
  breakatwhitespace=false,
  breaklines=true,
  captionpos=b,
  keepspaces=true,
  numbers=left,
  numbersep=5pt,
  showspaces=false,
  showstringspaces=false,
  showtabs=false,
  tabsize=2
}

\lstset{style=mystyle}
\usepackage{svg}

\usepackage{pgf,tikz}
\usetikzlibrary{arrows}
% \pagestyle{empty}
\definecolor{uququq}{rgb}{0.25,0.25,0.25}

% \usepackage[activate={true,nocompatibility},final,tracking=true,kerning=true,factor=1100,stretch=10,shrink=10]{microtype}

\newcommand{\boxalign}[2][0.986\textwidth]{
  \par\noindent\tikzstyle{mybox} = [draw=black,inner sep=6pt]
  \begin{center}\begin{tikzpicture}
   \node [mybox] (box){%
    \begin{minipage}{#1}{\vspace{-5mm}#2}\end{minipage}
   };
\end{tikzpicture}\end{center}}

\newcommand{\bigd}[2]{\frac{\mathrm{D} #1}{\mathrm{D} #2}}
\newcommand{\DD}{\mathrm{D}}

\author{Jacopo Tissino}

\allowdisplaybreaks