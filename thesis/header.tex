\usepackage[T1]{fontenc}
\usepackage[utf8]{inputenc}

% \renewcommand\mapstochar{\mathrel{\mathchoice%
% {\rlap{\rule[.05ex]{.1ex}{1ex}}\mkern-.5mu}%
% {\rlap{\rule[.05ex]{.1ex}{1ex}}\mkern-.5mu}%
% {\rlap{\rule[.035ex]{.08ex}{.75ex}}\mkern-.5mu}%
% {\rlap{\rule[.025ex]{.065ex}{.55ex}}\mkern-.5mu}%
% }}

% \renewcommand*\not{%
% \mathrel{%
%    \mathchoice%1
%       {\rlap{$\displaystyle\mkern2.5mu\mathnormal{/}$}}%
%       {\rlap{$\textstyle\mkern2.5mu\mathnormal{/}$}}%
%       {\rlap{$\scriptstyle\mkern2.5mu\mathnormal{/}$}}%
%       {\rlap{$\scriptscriptstyle\mkern2.5mu\mathnormal{/}$}}%
% }}

% \pdfminorversion=5
% \pdfobjcompresslevel=0

% \usepackage[a-3b]{pdfx}

\usepackage{indentfirst}
\usepackage{enumitem}
\usepackage{datetime2}
\usepackage{acronym}
% options (https://tex.stackexchange.com/questions/25520/how-can-i-use-the-latex-acronym-package-and-optionally-create-an-acronym-list-i):
% printonlyused: Only list used acronyms
% withpage: In printonlyused-mode show the page number where each acronym was first used.
% nolist: The option nolist stands for “don’t write the list of acronyms”.
% dua: The option dua stands for “don’t use acronyms”. It leads to a redefinition of \ac and \acp, making the full name appear all the time and suppressing all acronyms but the explicity requested by \acf or \acfp.

\makeatletter
\AtBeginDocument{%
  \renewcommand*{\AC@hyperlink}[2]{%
    \begingroup
      \hypersetup{hidelinks}%
      \hyperlink{#1}{#2}%
    \endgroup
  }%
}
\makeatother

\usepackage{algorithm}
\usepackage{algpseudocode}

\usepackage{textcomp}
\usepackage{multirow,bigdelim}
\usepackage{float}
% \usepackage[caption = false]{subfig}
% \usepackage{longtable}
% \usepackage{listings}
\usepackage{mathtools}
\DeclareMathOperator{\tr}{Tr}
\usepackage{commath}
% \usepackage{slashed}
% \usepackage{bbold}
\usepackage{xcolor}
\usepackage{physics}
% \newcommand{\lambdabar}{{\mkern0.75mu\mathchar '26\mkern -9.75mu\lambda}}
\usepackage[right=4cm,left=2cm,top=3cm,bottom=3.0cm, marginparwidth=2.7cm, marginparsep=3mm]{geometry}
\usepackage{mdframed}
\usepackage{amsmath}
% \usepackage{amsfonts}
\usepackage{amssymb}

% \numberwithin{equation}{section}
\usepackage{graphicx}

% \usepackage[colorinlistoftodos]{todonotes}
\PassOptionsToPackage{hyphens}{url}
\usepackage[colorlinks=true, allcolors=blue]{hyperref}
% \usepackage[pdfa]{hyperref}
\hypersetup{breaklinks=true}
% \hypersetup{colorlinks=true}
% \hypersetup{allcolors=blue}
% \usepackage{colorprofiles}

% \urlstyle{same}
\usepackage{siunitx}
\sisetup{separate-uncertainty=true}
% \DeclareSIUnit\parsec{pc}
\usepackage{cancel}
% \usepackage{mathrsfs}
\usepackage{marginnote}
\renewcommand*{\marginnotevadjust}{-0.3cm}
\renewcommand*{\marginfont}{\scriptsize}
% \usepackage{fancybox}

\usepackage{footnotebackref}

\usepackage[sc]{mathpazo}
\linespread{1.05}         % Palladio needs more leading (space between lines)

\newcommand\mybox[1]{%
  \fbox{\begin{minipage}{0.9\textwidth}#1\end{minipage}}}

\newcommand{\const}{\mathrm{const}}

\usepackage[section]{placeins}

\usepackage{tikz}
\usetikzlibrary{shapes,arrows,shadows}


\newcommand{\boxalign}[2][0.986\textwidth]{
  \par\noindent\tikzstyle{mybox} = [draw=black,inner sep=6pt]
  \begin{center}\begin{tikzpicture}
   \node [mybox] (box){%
    \begin{minipage}{#1}{\vspace{-5mm}#2}\end{minipage}
   };
\end{tikzpicture}\end{center}}

\pagestyle{plain}

\author{Jacopo Tissino}

\allowdisplaybreaks