\documentclass[main.tex]{subfiles}
\begin{document}

\section{MLGW-BNS}

Let us start with a bird's eye overview of the algorithm employed by \ac{mc} before getting into the technical details. 

We start with two systems for the generation of a theoretical waveform from a \ac{CBC} in the frequency domain, one being faster but less accurate than the other. 
The ``fast'' system will be the \ac{PN} model \texttt{TaylorF2} provided by \texttt{bajes} \cite[]{breschiTtBajesBayesian2021}, while the ``slow'' system will be the \ac{EOB} model \texttt{TEOBResumS}, with waveforms generated in the time domain and then Fourier-transformed. 

As we shall see shortly, a crucial feature of the ``fast'' system is the capability to generate waveforms in the frequency domain at select frequencies, without needing to perform a full Fourier transform from a time-domain waveform.
This feature is shared by all \ac{PN} approximants, which base their Fourier-domain representations on the analytic \ac{SPA}.

The main idea is then to 
\begin{enumerate}
    \item compute the residuals of the ``slow`` waveforms from the ``fast`` ones, and with these build a dataset;
    \item train a machine learning system on this dataset, so that it is able to reconstruct the map from the parameters \(\vec{\theta}\) to the residuals.
\end{enumerate}

If this task is accomplished with tolerable errors and with a fast enough execution time (shorter than the time taken by the ``slow'' generation method) we will have a working \emph{surrogate model}. 

\subsection{Waveform generation and downsampling}

\subsubsection{Greedy downsampling}

The residual waveforms generated with the \ac{EOB} / \ac{PN} models have on the order of a few times \num{e5} sampling points. 
This is too large a number to effectively use \ac{PCA} on: as we will see, it requires writing a covariance matrix which would have \(\sim \num{e11}\) entries, way beyond the \ac{RAM} of most computers. 

Therefore, we need to start by applying a simple dimensionality reduction technique: downsampling, applied to the unwrapped phase and amplitude of waveforms (see section \ref{sec:unwrapping}).

Doing so on a uniform grid, however, is suboptimal: 

\textcite{galleyFastEfficientEvaluation2016} introduced \texttt{romspline}. 


\subsection{Phase unwrapping} \label{sec:unwrapping}

\subsection{Residual calculation} \label{sec:residuals}



\subsection{}

% The \ac{mb} algorithm is an attempt at accelerating waveform generation through interpolation of slow-to-generate waveforms. 

% The ingredients for 

% Things I do to the waveform:
% \begin{enumerate}
%     \item Calculate the Fourier-transformed \(h _{\text{EOB}}(f, \theta ) = A _{\text{EOB}} (\theta ) e^{i \phi _{\text{EOB}} (f, \theta )}\)
%     \item calculate \(h _{\text{PN}} (f, \theta ) = A _{\text{PN}} e^{i \phi _{\text{PN}} (f, \theta )}\)
% \end{enumerate}


\end{document}