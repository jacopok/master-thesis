\documentclass[main.tex]{subfiles}
\begin{document}


\topskip0pt
\vspace*{\fill}
\section*{Abstract}
The analysis of interferometric data corresponding to the gravitational waves emitted in a binary neutron star merger requires the generation of numerous theoretical waveforms.
In this thesis I develop a machine learning algorithm trained on frequency-domain waveforms generated through the effective one-body technique, which is able to reconstruct them accurately and with a speed improvement.
\vspace*{\fill}


\newpage

\tableofcontents

% \subsection*{Conventions and notation}

% \(c=1\) for sure.

% Tensors with mixed indices should be written as \(A^{\mu }{}_{\nu }\), but if \(A_{\mu \nu }\) is symmetric then \(A^{\mu }{}_\nu = A_\nu {}^{\mu }\). 

\newpage

\section*{Introduction}

% mark the some acronyms as used --- I only want the short form
\acused{mb}
\acused{LIGO}
\acused{SI}

\paragraph{A brief history of GW astronomy}

The existence of \acp{GW} was first postulated by Einstein in 1916 \cites{1916SPAW.......688E}{1918SPAW.......154E}, just one year after he had published the theory of \ac{GR} \cite[]{einsteinFeldgleichungenGravitation1915}. 
For several decades it was debated whether they were even a true prediction of the theory or simply a gauge artifact \cite{kennefickTravelingSpeedThought2007}. 

In the second half of the 20th century efforts started towards the creation of instruments which would be able to directly detect these waves. 
The sources which, it was thought, would be easiest to detect were binaries of compact objects: \acp{BH} and \acp{NS}.

It was known that these binary systems existed, and that their orbits shrunk thanks to \ac{GW} emission --- a pulsar in a binary system had been detected in the 1970s and the decay of its orbital period had been measured to be precisely compatible with the \ac{GR} prediction \cites{hulseDiscoveryPulsarBinary1975}{taylorNewTestGeneral1982}.
Theoretical results indicated that they would emit a very loud signal when they finally merged, and that this signal would have a specific chirping shape. 

On the 14th of September 2015 the first \ac{GW} signal from a \ac{BBH} system was detected \cite[]{ligoscientificcollaborationandvirgocollaborationObservationGravitationalWaves2016} by the \ac{LIGO} interferometers in the United States. 
Two years later, on the 17th of August 2017, the fist \ac{GW} signal from a \ac{BNS} merger \cite{abbottGW170817ObservationGravitational2017}
was detected, and at that point the interferometer Virgo in Italy had joined the collaboration.
These events have been dubbed GW150914 and GW170817 --- these are their entries in what is now a growing, publicly available\footnote{See \url{https://www.gw-openscience.org} for a list of events or \url{https://catalog.cardiffgravity.org/} for a graphical representation of their estimated parameters.} database of observations, containing several dozen signals from \acp{CBC}.

The new field of \ac{GW} astronomy has already provided several interesting results, such as unprecedented tests of strong-field gravity, as well an exploration of astrophysical populations of stellar-mass compact objects and their properties: masses, spins, and --- for \acp{NS} --- tidal deformabilities. 

This, however, comes with several hard challenges in terms of, among other things, data analysis and signal modelling.
Even with the best insulation available from the environment, \acp{GW} are so faint that they arrive buried in noise. 
Clever data analysis strategies and a lot of computational power are needed to detect the presence, and even more so to study the properties of the systems which generated them.

\paragraph{Waveform generation}

The study of an observed signal is done through Bayesian parameter estimation algorithms, which require the comparison of the experimental signal with several million \cite{lackeyEffectiveonebodyWaveformsBinary2017} simulated waveforms or more (see section \ref{sec:data-analysis}).
Speed in the generation of simulated waveforms is, therefore, crucial.

There are three main strategies for the generation of these waveforms. In order of (typically) decreasing evaluation time, as well as decreasing accuracy, they are:
\begin{enumerate}
    \item full \ac{NR} simulations: these are our most accurate possible avenue for the study of \ac{BNS} mergers, but they are so computationally expensive that it is currently only possible to simulate at most a few tens of initial condition setups --- these are then used to validate or inform the other methods (see section \ref{sec:nr});
    \item \ac{EOB} simulations: these consist in constructing an effective Hamiltonian for a test particle moving in an effective metric as a proxy for the two-body system and simulating its evolution --- they can be calibrated by comparison to \ac{NR} simulations (see section \ref{sec:eob});
    \item \ac{PN} waveforms: these are analytical solutions to the equations of motion of the system, expanded up to a certain order in powers of $1/c^2$ (see section \ref{sec:post-newtonian}).
\end{enumerate}

Both the \ac{EOB} and \ac{PN} formalisms can be complemented with a tidal term, which is relevant for \ac{BNS} mergers: neutron stars (as opposed to black holes) can be deformed by each other's gravitational field in a way which depends on the specifics of their equation of state (see section \ref{sec:cbc-parameters}). 

Currently, only \ac{EOB} systems are accurate enough not to bias parameter estimation (see section \ref{sec:target-fidelity}) while being fast enough to be practically useful.
The evaluation times for \ac{EOB} waveforms are dependent on the duration of the waveforms: GW170817 lasted for about two minutes, and the generation of waveforms of this length takes on the order of $100\text{ms}$ with state-of-the-art \ac{EOB} systems, but as the noise in ground-based interferometers decreases we expect to be able to see a longer and longer section of the waveform, which might bring the evaluation times back above $1\text{s}$.

Also, the waveforms ought to be generated in the frequency domain, since that form is the one which is used in data analysis;
EOB models, on the other hand, natively generate waveforms in the time domain. 

\paragraph{The algorithm}

The system developed in this thesis, \ac{mb}, is an attempt to make progress in this direction, by developing a machine learning system trained on the (Fourier transforms of the) \ac{EOB} waveforms for \ac{BNS}. 
This builds on the work of \textcite{schmidtMachineLearningGravitational2020}, who developed a similar system to generate \ac{BBH} merger waveforms in the time domain.

The algorithm's basic components are explained in detail in chapter \ref{chap:mlgw-bns} and summarized here (see also the diagram in figure \ref{fig:flowchart}):
\begin{enumerate}
    \item a training dataset is generated by considering uniformly distributed tuples of \ac{BNS} parameters in the allowed ranges and calculating the corresponding waveforms with the \ac{EOB} model \texttt{TEOBResumS} \cite[]{nagarTimedomainEffectiveonebodyGravitational2018};
    \item each waveform is approximately Fourier-transformed with a \ac{SPA} \cite[]{gambaFastFaithfulFrequencydomain2020} and decomposed into phase and amplitude;
    \item each \ac{EOB} waveform is compared to a \ac{PN} one with the same parameters, and a dataset is constructed from the residuals of the \ac{EOB} waveform from the \ac{PN} one;
    \item the dimensionality of the training dataset is reduced through \ac{PCA};
    \item a feed-forward neural network is trained to reconstruct the map between the \ac{BNS} system parameters and the \ac{PCA} components of the waveforms. 
\end{enumerate}

The resulting model must be evaluated in terms of both speed and accuracy.

Regarding \textbf{speed}, the main bottleneck is the same as for \ac{EOB} models: interpolating the waveform to a uniform frequency grid. 
Because of this, in standard benchmarks there is not a significant improvement (see section \ref{sec:evaluation-time}). 
However, data analysis for future interferometers will need to make use of reduced-order modelling; in that context, realistic values for the  speedup are of an order of magnitude or more (see section \ref{sec:downsampled-evaluation}). 

Regarding \textbf{accuracy}, the model is able to reconstruct \ac{EOB} waveforms with five intrinsic parameters (the mass ratio \(q\), the two tidal deformabilities \(\Lambda_1\) and \(\Lambda_2 \), the two aligned spin components \(\chi_1 \) and \(\chi_2 \)) with a maximal unfaithfulness of the order of \(\mathcal{F} \simeq \num{e-4}\) (see section \ref{sec:accuracy}) using less than 1000 training waveforms.
This is lower than the typical values of the unfaithfulness of EOB waveforms compared to \ac{NR} ones (\(\mathcal{F} \simeq \num{2.5e-3}\) \cite{nagarTimedomainEffectiveonebodyGravitational2018}), and is acceptable for the parameter-estimation needs of current detectors (see section \ref{sec:target-fidelity}).

If more training data is used, the fidelity of \ac{mb} to its reference model steadily improves; with 50 thousand training waveforms it reaches \(\mathcal{F} \lesssim \num{e-6}\) (see figure \ref{fig:mismatches_by_n_train}). 

This indicates that it is possible for this system to become a part of a gravitational wave data analysis pipeline for future interferometers such as \ac{ET}, enabling a significant speedup of an accurate underlying model at the cost of a negligible loss in accuracy. 

% Waveforms are needed for both parameter estimation and modelled signal searches
% Possible applications of the software developed in this work include modelled signal searches and parameter estimation for both current interferometers (such as LIGO and Virgo) and, with due improvements, for future ones such as Einstein Telescope.

\paragraph{This thesis}

This thesis starts with a discussion of the basics of gravitational wave theory in general (\ref{sec:linearized-gravity}) and in the context of a \ac{CBC} (\ref{sec:compact-binaries-linear}).
We then discuss how a waveform from a \ac{CBC} should be parametrized in the context of parameter estimation (\ref{sec:cbc-parameters}), and the relevant data analysis methods (\ref{sec:data-analysis}, \ref{sec:bayesian}). 

We give a brief rundown on the \ac{PN}, \ac{EOB} and \ac{NR} methods for the generation of waveforms in section \ref{sec:higher-order-waveforms}, and then in chapter \ref{chap:ml} we outline the main \ac{ML} methods needed for \ac{mb}: greedy downsampling (\ref{sec:downsampling}), \ac{PCA} (\ref{sec:principal-component-analysis}), \acp{NN} (\ref{sec:neural-network}). 

Finally, chapter \ref{chap:mlgw-bns} outlines the internal workings of the \ac{mb} model. 

\end{document}