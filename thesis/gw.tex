\documentclass[main.tex]{subfiles}
\begin{document}

\section{Gravitational wave theory}

\subsection{Linearized gravity}

The simplest way to discuss gravitational radiation is to consider linearized gravity on a flat Minkowskian background. 

We give a brief overview, roughly following the path taken by \textcite[chapter 1]{maggioreGravitationalWavesVolume2007}.
 
This means that we assume that our spacetime admits a reference frame for which the metric is in the form 
%
\begin{align}
g_{\mu \nu } = \eta_{\mu \nu } + h_{\mu \nu }
\,,
\end{align}
%
where the value of the components \(h_{\mu \nu }\) is small enough that we can work to first order in them. 
Any equation in this section includes an implicit ``\(+ \order{h^2}\)''.
We work in this \emph{global inertial frame}. 

In order to study the evolution of the perturbation \(h_{\mu \nu }\) we need to solve the Einstein Field Equations for it to linear order. 
In a vacuum, they can be written as 
%
\begin{align}
G_{\mu \nu } &= R_{\mu \nu} - \frac{1}{2} \eta_{\mu \nu } R  = 0 \\
R_{\mu \nu } &= g^{\alpha \beta } R_{\alpha \mu \beta  \nu }   \\
R &= g^{\mu \nu } R_{\mu \nu }
\,,
\end{align}
%
where \(R_{\alpha \mu \beta \nu } \sim \partial \Gamma + \Gamma \Gamma \) is the Riemann tensor, which is written in terms of derivatives and squares of Christoffel symbols \(\Gamma \): 
%
\begin{align}
\Gamma^{\rho }_{\mu \nu } 
&= \frac{1}{2} g^{\rho \lambda } 
\qty( \partial_{\mu } g_{\nu \lambda } + \partial_{\nu } g_{\lambda \mu } - \partial_{\lambda } g_{\mu \nu }) \\
&= \frac{1}{2} \eta ^{\rho \lambda } 
\qty( \partial_{\mu } h_{\nu \lambda } + \partial_{\nu } h_{\lambda \mu } - \partial_{\lambda } h_{\mu \nu }) 
\marginnote{We use the fact that \(\partial \eta = 0\), and keep only linear order terms.}
\,.
\end{align}

Since the Christoffel symbols are of first order in the perturbation, the term \(\Gamma \Gamma \) in the Riemann tensor is of second order and can be neglected. 
Therefore, the relevant components are 
%
\begin{align}
R_{\mu \nu \rho \sigma } = 2\eta_{\mu \lambda } \Gamma^{\lambda }_{\nu [\sigma , \rho ]}  = h_{\mu [\sigma , \rho] \nu } 
- h_{\nu [\sigma , \rho ] \mu  }
\,,
\end{align}
%
which gives us the following expression for the Ricci tensor: 
%
\begin{align}
R_{\mu \nu } = \frac{1}{2}
\qty( 
    h^{\sigma }{}_{\mu, \sigma \nu  } +
    h^{\sigma }{}_{\nu , \sigma \mu } - 
    h_{, \mu \nu  } -  
    \square h_{\mu \nu } 
)
\,,
\end{align}
%
where \(h = \eta^{\mu \nu } h_{\mu \nu }\) is the trace of the perturbation (computed with respect to the flat metric), while \(\square = \eta^{\mu \nu } \partial_{\mu } \partial_{\nu }\) is the flat space d'Alambertian. 

This, in turn, allows us to write out the Einstein tensor: 
%
\begin{align}
G_{\mu \nu } = \frac{1}{2} \qty(
    h^{\sigma }{}_{\mu, \sigma \nu  } +
    h^{\sigma }{}_{\nu , \sigma \mu } - 
    h_{, \mu \nu  } -  
    \square h_{\mu \nu } -
    \eta_{\mu \nu } h^{\rho \lambda }{}_{, \rho \lambda } + \eta_{\mu \nu } \square h
)
\,.
\end{align}

This can be greatly simplified with two steps:
first, we change variable from the perturbation \(h_{\mu \nu }\) to the \emph{trace-reversed} perturbation \(\overline{h}_{\mu \nu } = h_{\mu \nu } - \eta_{\mu \nu } h / 2\) --- the name comes from the fact that \(\eta^{\mu \nu } \overline{h}_{\mu \nu } = - h\).

This substitution allows us to write the Einstein tensor as 
%
\begin{align}
G_{\mu \nu } = - \frac{1}{2} \square \overline{h}_{\mu \nu } +  \overline{h}_{\alpha (\mu , \nu )}{}^{\alpha } 
- \frac{1}{2} \eta_{\mu \nu } \overline{h}_{\alpha \beta }{}^{, \alpha \beta }
\,.
\end{align}

We will shortly show that it is possible, as a \emph{gauge choice}, to set the divergence of the trace-reversed perturbation to zero: \(\partial^{\mu } \overline{h}_{\mu \nu }= 0\) (see section \ref{sec:gauge-fixing}). The gauge imposed by this choice is called the \emph{Hilbert Gauge},\footnote{Despite the name, this choice was first suggested by De Sitter to Einstein \cite{kennefickTravelingSpeedThought2007}, who had been previously trying to impose the gauge \(\abs{g}= 1\) \cite[page 688]{1916SPAW.......688E}.} which in terms of the regular perturbation reads 
%
\begin{align} \label{eq:hilbert-gauge}
\partial^{\mu } h_{\mu \nu } - \frac{1}{2} \partial_{\nu } h = 0
\,.
\end{align}

With this choice the Einstein tensor becomes simply 
%
\begin{align}
G_{\mu \nu } = - \frac{1}{2} \square \overline{h}_{\mu \nu }
\,,
\end{align}
%
so the general form of the Einstein equations to linear order will be 
%
%
\boxalign{
\begin{align} \label{eq:wave-equation-einstein}
\square \overline{h}_{\mu \nu }= - 16 \pi G \eval{T_{\mu \nu }}_{\text{linear}}
\,,
\end{align}}
%
where the stress-energy tensor is computed up to first order in the metric perturbation.

\subsubsection{Transformations of the perturbation}

The theory of General Relativity is constructed to be invariant under smooth changes of coordinates: under a map in the form \(x \to x' = x' (x)\) (where \(x'(x)\) is a diffeomorphism\footnote{In the physics parlance this property is known as ``diffeomorphism invariance'', while a mathematician would call the kinds of transformations considered ``isometries'', since we ask that they preserve the metric structure of the manifold.}).

Under such a coordinate transformation the metric transforms like any \((0, 2)\) tensor:
%
\begin{align} \label{eq:metric-transformation-general}
g_{\alpha \beta }' ( x' ) = \pdv{x^{ \mu }}{x^{\prime \alpha }} 
\pdv{x^{ \nu }}{x^{\prime \beta }} 
g_{\mu \nu } ( x)
\,.
\end{align}

A general transformation of this kind may break the condition that \(g = \eta + h\) where \(h\) is small, so in order to preserve our framework we restrict ourselves to a small class of transformations. 

One possibility is to consider infinitesimal transformations in the form 
%
\begin{align}
x^{\mu } \to x^{\prime \mu } = x^{\mu } + \xi^{\mu } (x)
\,,
\end{align}
%
where \(\xi^{\mu }\) is a vector field such that \(\abs{\partial_{\mu } \xi_{\nu }}\) is small --- specifically, the condition to impose is that the first order in \(\partial_{\mu } \xi_{\nu }\) should match the first order in \(h_{\mu \nu }\). 

This condition is all we need in order to write the transformation law for the perturbation: the full equation reads
%
\begin{align}
\eta_{\mu \nu } ' + h_{\mu \nu }' \approx 
\qty(\delta^{\alpha }_{\mu } - \partial^{\alpha } \xi_\mu ) 
\qty(\delta^{\beta  }_{\nu  } - \partial^{\beta  } \xi_\nu  ) 
\qty(\eta_{\alpha \beta } + h_{\mu \nu })
\marginnote{Used \(1 / (1+x) = 1-x + \order{x^2}\).}
\,,
\end{align}
%
so the zeroth order contribution is \(\eta'_{\mu \nu } = \eta_{\mu \nu }\), while the first order one is 
%

%
\boxalign{
\begin{align} \label{eq:perturbation-transformation-infinitesimal}
h^{\prime }_{\mu \nu } = h_{\mu \nu } - 2 \partial_{(\mu } \xi_{\nu )}
\,,
\end{align}}
%
which is our transformation law for the metric perturbation. 

We will also need a transformation law for the trace-reversed perturbation \(\overline{h}_{\mu \nu }\): the trace transforms as \(h' \to h - 2 \partial_{\mu } \xi^{\mu }\), therefore the required law is 
%
\begin{align} \label{eq:perturbation-transformation-infinitesimal-tracereversed}
\overline{h}'_{\mu \nu } = \overline{h}_{\mu \nu } - 2 \partial_{(\mu } \xi_{\nu )} + \eta_{\mu \nu } \partial_{\alpha } \xi^{\alpha }
\,,
\end{align}
%


A second class of transformations is a subset of Lorentz boosts and rotations: substituting \(\pdv*{x^{\mu }}{x^{\prime \nu }} = \Lambda_\nu{}^{\mu }\) into the transformation law \eqref{eq:metric-transformation-general} we find that the flat metric is unchanged, while 
%
\begin{align}
h^{\prime }_{\mu \nu } ( x') = \Lambda_{\mu }{}^{\alpha }
\Lambda_{\nu }{}^{\beta } h_{\alpha \beta }
\,,
\end{align}
%
which may remain in the class of small metric perturbations: this is not guaranteed, but it is true for a certain subset of boosts and for all rotations \cite{maggioreGravitationalWavesVolume2007}. 

Finally, the perturbation is invariant under shifts in the form \(x^{\prime \mu }= x^{\mu } + a^{\mu }\). 

\subsubsection{Gauge fixing} \label{sec:gauge-fixing}

Now that we know how the perturbation \(h_{\mu \nu }\) transforms under an infinitesimal transformation, we can use this to impose the condition we want --- specifically, the Hilbert gauge \eqref{eq:hilbert-gauge}.

The way to show that this is possible is to write out the way \(\partial^{\mu } \overline{h}_{\mu \nu }\) transforms for an arbitrary choice of \(\xi \), and to see that with an appropriate choice of \(\xi \) we can always map it to zero. 
The transformation reads 
%
\begin{align}
\partial^{\mu } \overline{h}_{\mu \nu }' &= 
\partial^{\mu } \qty(h_{\mu \nu } - 2 \partial_{(\mu } \xi_{\nu )}) - \frac{1}{2} \partial_{\nu } \qty(\eta^{\alpha \beta } \qty(h_{\alpha \beta } - 2 \partial_{(\alpha } \xi_{\beta )}))  \\
&= \partial^{\mu } \overline{h}_{\mu \nu } - \square \xi_{\nu } - \partial_{\nu } \qty(\partial^{\mu } \xi_{\mu }) + \partial_{\nu } \qty(\partial^{\mu } \xi_{\mu })  \\
&= \partial^{\mu } \overline{h}_{\mu \nu } - \square \xi_{\nu }
\,. 
\end{align}

Therefore, from any starting gauge we must only find a \(\xi_{\nu }\) such that \(\partial^{\mu } \overline{h}_{\mu \nu } = \square \xi_{\nu }\), and we will be in the correct gauge. 
This can always be done, since the D'Alambert equation \(\square f = g\) can always be solved for \(f\) --- if we needed to compute \(\xi_{\nu }\) explicitly (which we typically do not) we could use the Green's function \(G(z)\) for the operator, defined by \(\square G(z) = \delta^{(4)} (z)\). 

While the equation is solvable, the solution is not unique: if we were to define an ``inverse'' of the D'Alambertian it would not be a function but a one-to-many relation. 
Specifically, while keeping fixed the value of \(\square \xi_{\nu }\) we can add any function \(\zeta_\nu \) to \(\xi _\nu  \) as long as \(\square \zeta _\nu = 0 \). 
A trivial example is \(\zeta _\nu = \const\), but other wave-like choices are of more interest. 

These still induce a transformation on \(h_{\mu \nu }\) according to the usual law \eqref{eq:perturbation-transformation-infinitesimal}, and they can be used to further specify the form of the gravitational radiation. 

In terms of \textbf{degrees of freedom}, the full perturbation \(h_{\mu \nu }\) starts with 10 as any symmetric 4D, rank-2 tensor; the four Hilbert gauge conditions \eqref{eq:hilbert-gauge} reduce them to 6, while the four residual gauge conditions will allow us to reduce them to 2. 

Doing so in full generality is not useful for us, let us instead fix the residual gauge in the specific context of a plane wave solution to the wave equation \(\square \overline{h}_{\mu \nu } = 0\).

\subsubsection{Plane gravitational waves}

From an analogy to electromagnetic theory it seems reasonable to work with an ansatz in the form 
%
\begin{align}
\overline{h}_{\mu \nu } = A_{\mu \nu } e^{i k_\alpha x^{\alpha }}
\,,
\end{align}
%
where \(A_{\mu \nu }\) is a constant symmetric tensor.

Imposing the wave equation sets \(k_{\alpha } k^{\alpha } = 0\), and the Hilbert gauge \eqref{eq:hilbert-gauge} condition can be written as \(A_{\mu \nu } k^{\mu } = 0\), where \(A = \eta^{\mu \nu } A_{\mu \nu }\). 

In this framework we can impose the residual gauge condition explicitly: a function which satisfies \(\square \zeta_{\mu } = 0\) is \(\zeta_{\mu } = B_{\mu } \exp(i d_\alpha x^{\alpha })\), where \(d_\alpha \) is a null vector (\(d_\alpha d^{\alpha } = 0\)) while \(B_\mu \) is a generic constant vector.

In these terms, the transformation equation \eqref{eq:perturbation-transformation-infinitesimal-tracereversed} reads 
%
\begin{align}
\overline{h}_{\mu \nu } &\to \overline{h}_{\mu \nu } + \qty(- 2 i B_{(\mu } d_{\nu )}  + i \eta_{\mu \nu } B_{\beta } d^{\beta } ) e^{i d_\alpha x^{\alpha }} \\
A_{\mu \nu } e^{i k_\alpha x^{\alpha }} &\to A_{\mu \nu } e^{i k_\alpha x^{\alpha }} + \qty(-2 i B_{(\mu } d_{\nu )} + i \eta_{\mu \nu } B_{\beta } d^{\beta }) e^{i d_\alpha x^{\alpha }} 
\,.
\end{align}

This tells us that if we set the vector \(d_{\alpha }\) to be equal to \(k_\alpha \) the amplitude \(A_{\mu \nu }\) will transform according to the algebraic system
%
\begin{align}
A_{\mu \nu } \to A_{\mu \nu } - 2 i B_{(\mu } k_{\nu )}
\,,
\end{align}
%
which allows us to impose four conditions on \(A_{\mu \nu }\), one for each of the free components of \(B_{\mu }\). 
It is customary to choose \(A = 0 = A_{0 i }\): these are known together as the \ac{TT} gauge. 

The condition \(A = 0\) also means that \(h = 0 = \overline{h}\): thus, form this point onward we can stop distinguishing between \(h_{\mu \nu }\) and  \(\overline{h}_{\mu \nu }\), and for simplicity's sake we write the former. 

If we orient our axes such that \(\vec{k} = k^{i}\) is along the \(\hat{z}\) direction (which means \(k^{\mu } = (k, 0, 0, k)^{\top}\)) the conditions can be written as 
\begin{enumerate}
    \item Hilbert gauge + traceless: \(A_{\mu 0}+ A_{\mu 3} = 0\);
    \item traceless: \(A = - A_{00} + A_{11} + A_{22} + A_{33} = 0 \);
    \item transverse: \(A_{0i} = 0\). 
\end{enumerate}

The Hilbert gauge combined with the transverse conditions show that \(A_{00} = 0\) as well, followed by \(A_{13} = 0\), \(A_{23} = 0\) and \(A_{33} = 0\). 

Finally, the traceless condition imposes \(A_{11} = - A_{22} \).
These conditions tell us that the plane gravitational wave must have the form 
%

%
\boxalign{
\begin{align} \label{eq:TT-gauge-gw}
h_{\mu \nu }(x) = \left[\begin{array}{cccc}
0 & 0 & 0 & 0 \\ 
0 & h_+ & h_\times  & 0 \\ 
0 & h_\times  & - h_+ & 0 \\ 
0 & 0 & 0 & 0
\end{array}\right]
\exp(i k_{\alpha } x^{ \alpha })
\,,
\end{align}}
%
where \(h_+ = A_{11} \) and \(h_\times = A_{12} \). 

\subsubsection{Effect on test masses}

The \ac{TT} gauge we defined has the rather peculiar characteristic of ``moving with the wave'', so that the position of any observer initially at rest (so, such that \(u^{\mu } (\tau = 0 ) = (1, \vec{0})^{\top}\)) is unchanged: the geodesic equation evaluated at \(\tau = 0\) reads
%
\begin{align}
\eval{\dv[2]{x^{\mu }}{\tau}}_{\tau = 0} + \eval{\Gamma^{\mu }_{\nu \rho } u^{\nu } u^{\rho }}_{\tau = 0} &= 0   \\
\eval{\dv[2]{x^{\mu }}{\tau}}_{\tau = 0} &= - \Gamma^{\mu }_{00} = 0 
\,,
\end{align}
%
since the Christoffel symbols computed \(\Gamma^{\mu }_{00}\) with the \ac{TT} gauge perturbation \eqref{eq:TT-gauge-gw} all vanish: \(\Gamma^{\mu }_{00} = \eta^{\mu \nu } \qty( 2 g_{\nu 0,0} - g_{00, \nu }) / 2 =  0\).

Does this mean that gravitational waves are merely an artefact, and have no effect on particles? 
No, since while in the \ac{TT} gauge the \emph{positions} of the points do not change, the \emph{distance} among them does.

In order to understand this effect we can make use of the geodesic deviation equation \cite[section 3.10]{carrollSpacetimeGeometryIntroduction2019}, which states that if we take two geodesics whose four-velocities are both approximately \(u^{\mu }\), separated by a short vector \(\xi ^{\mu }\),\footnote{There is a technical note to be made here: in GR the notion of a vector between two points in the manifold is meaningless, however we can get around this problem by considering a one-parameter family of geodesics, and identifying the separation vector between them to be the tangent vector associated to the parameter.} they might diverge or converge, and the evolution of \(\xi^{\mu }\) will be described by
%
\begin{align}
\ddot{\xi}^{\mu } = R^{\mu }_{\nu \rho \sigma } u^{\nu } u^{\rho } \xi^{\sigma }
\,.
\end{align}

Let us consider the same geodesics as before, whose four-velocity is uniformly \(u^{\mu } \equiv \qty(1, \vec{0})^{\top}\): the ``acceleration'' will then be given by the matrix product \(R^{\mu }_{00\sigma } \xi^{\sigma }\). 

These components of the Riemann tensor read \cite[eq.\ 7.106]{carrollSpacetimeGeometryIntroduction2019}: 
%
\begin{align}
R^{\mu }_{00 \sigma } = \frac{1}{2} \ddot{h}^{\mu }_{\sigma }
\,.
\end{align}

Therefore, the temporal component of the acceleration is \(\ddot{\xi}^{0} \propto \ddot{h}^{0}_{\sigma } = 0 \), while the spatial components read 
%
\begin{align} \label{eq:tidal-acceleration-TT-gauge}
\ddot{\xi}
^{i} = \frac{1}{2} \ddot{h}^{i}_{j} \xi^{j}
\,.
\end{align}

These equations can then be explicitly solved.
It is important to remark that \(\ddot{\xi }^{i}\) is \emph{not} coordinate acceleration: the points are \emph{stationary} in the \ac{TT} gauge coordinates, but the distance among them changes.
Intuitively, we can interpret this as due to our carefully constructed gauge choice, which was determined through residual gauge, so it satisfied a wave equation \(\square \zeta _\mu =0 \): it precisely oscillates with the wave, allowing the position of any point to not change (to linear order). 
Fortunately \acsp{GW} are physical, so they must still manifest regardless of our gauge, and they do so through the distortion of distances as described here.

\begin{figure}[ht]
\centering
\includegraphics[width=.7\textwidth]{figures/polarizations}
\caption{Polarizations of gravitational waves. Time evolution is represented though transparency: darker ellipses correspond to later points in time. The four cases are described in the text. We stress that in the \ac{TT} gauge the ellipses do not represent changes in coordinate position but only in distance among points. In all cases the starting configuration is circular. The code generating this figure is available \href{https://github.com/jacopok/master-thesis/blob/main/thesis/figures/polarizations.py}{here}.}
\label{fig:polarizations}
\end{figure}

The positions of points at a fixed position (\(x\), \(y\), \(z\)) change (up to a phase) according to the expression 
%
\begin{align}
x (t) &= x(t=0) \qty(1 + h_+ e^{i \omega t}) + y(t=0) h_\times  e^{i \omega t} \\
y (t) &= y(t=0) \qty(1 - h_+ e^{i \omega t}) + x(t=0) h_\times  e^{i \omega t} 
\,,
\end{align}
%
which is shown graphically in figure \ref{fig:polarizations} for four configurations: only \(h_+\) or \(h_\times \) being nonzero, or only \(h_R \propto h_+ + i h_\times \) or \(h_L \propto h_+ - i h_\times \) being nonzero. 
The last two are called \emph{circular polarizations}.

This description is not the one an experimentalist might use: if we need to consider the noise due to other sources of gravitation, it is convenient to move from the \ac{TT} gauge to the \emph{proper detector frame}, in which the metric is expanded around a fiducial point of our detector --- for instance, around the beam-splitter in an interferometer.
In such a frame, it can be shown that the effect of \acsp{GW} can be described as a Newtonian force on points a certain distance away from it --- for instance, the mirrors of the interferometer. This comes in at second order in the distance, and it is given by \cite[eq.\ 1.96]{maggioreGravitationalWavesVolume2007}
%
\begin{align}
F_{i} = \frac{m}{2} \ddot{h}_{ij}^{TT} \xi^{j}
\,,
\end{align}
%
which unsurprisingly closely mirrors the ac{TT} gauge variation of the distance: the physically measurable effect must be independent of gauge choice. 
The reason the \ac{TT} gauge perturbation appears here as well is that the Riemann tensor, which is used in these computations, is \emph{invariant} (a stronger condition than covariant! the values of the components are actually unchanged) under gauge transformations. 

\subsubsection{The quadrupole formula}

The lowest order contribution to the generation of gravitational waves can be calculated starting from the linearized Einstein equation \eqref{eq:wave-equation-einstein}. 
The Green's function method for the inversion of the D'Alambertian is a common technique: if we can find a function \(G(z)\) such that \(\square G(z) = \delta^{(4)} (z)\), then we can do the following manipulation (denoting the gravitational constant as \(G_N\) for clarity): 
%
\begin{align}
\square_x \overline{h}_{\mu \nu} (x) &= - 16 \pi G_N T_{\mu \nu } (x)  \\
&= - 16 \pi G_N \int \dd[4]{y} T_{\mu \nu }(y) \delta^{(4)} (x-y)  \\
&= - 16 \pi G_N \int \dd[4]{y} T_{\mu \nu }(y) \square_x G(x-y)  \\
&= \square_x \qty(- 16 \pi G_N \int \dd[4]{y} T_{\mu \nu }(y) G(x-y))  
\marginnote{Removing the D'Alambertian is valid insamuch as it will give us \emph{a} solution for the wave equation, which will not be unique:  \(\square f = \square g\) does not imply \(f = g\), but the reverse is true.} 
\\
\overline{h}_{\mu \nu } &= -16 \pi G_N \int \dd[4]{y} T_{\mu \nu }(y) G(x-y)  
\\
&= 4 G_N \int \dd[3]{y} T_{\mu \nu }(x^{0} - \abs{\vec{x} - \vec{y}}, \vec{y}) \frac{1}{\abs{\vec{x} - \vec{y}}}
\,,
\end{align}
%
where in the last step we introduced the explicit expression for the Green's function of the D'Alambertian: \(G(z) = -[z^{0}>0] / 4 \pi \abs{\vec{z}}\)\footnote{The square bracket here is the Iverson bracket \cite{knuthTwoNotesNotation1992}, which maps a boolean expression to 1 (if true) or 0 (if false); in this context it is thus equivalent to the Heaviside Theta: \([z^{0}>0] = \Theta (z^{0})\).} \cite[eq.\ 3.6]{maggioreGravitationalWavesVolume2007}. 

We then have a formula for the trace-reversed perturbation, and it can be shown \cite[page 10]{maggioreGravitationalWavesVolume2007} that far from the source we can recover the \ac{TT} gauge perturbation by projecting this equation thanks to a tensor \(\Lambda_{ij, kl} (\hat{n})\): 
%
\begin{align}
h_{ij}^{TT} (t, \vec{x}) = 4G_N \Lambda_{ij, kl}(\hat{n}) \int \frac{\dd[3]{y}}{\abs{\vec{x} - \vec{y}}} T_{kl} \qty(t - \abs{\vec{x} - \vec{y}}, \vec{y}) 
\,,
\end{align}
%
where the tensor \(\Lambda \) is constructed to be a projector which sends 2-tensors into the subspace of traceless tensors, which are transverse to the direction defined by \(\hat{n} = \vec{x} / \abs{\vec{x}}\): 
%
\begin{align}
\Lambda_{ij, kl} = P_{ik} P_{jl}- \frac{1}{2} P_{ij} P_{kl} 
\qquad \text{where} \qquad
P_{ ij} = \delta_{ij} - n_i n_j   
\,.
\end{align}

This expression can be further simplified by making use of the assumption that the source is far away: the integral over \(\dd[3]{y}\) ranges over a region which has a maximum size of \(R\), the scale of the source, while the wave is observed at a distance \(r = \abs{\vec{x}} \gg R\). 
This allows us to expand and neglect terms of the order \(R^2 / r^2\) and over;\footnote{
    Specifically, we expand 
    %
    \begin{align}
    \abs{\vec{x} - \vec{y}} = \sqrt{x^2 + y^2 + 2 x \cdot y} 
    = r \sqrt{1 - 2 \frac{\hat{n} \cdot \vec{y}}{r} + \frac{y^2}{r^2}}
    \approx r \qty(1 - 2 \frac{\hat{n} \cdot \vec{y}}{r} + \order{R^2/ r^2})
    \,.
    \end{align}
} the result we get is 
%
\begin{align}
h_{ij}^{TT} = \frac{4 G_N}{r} \Lambda_{ij, kl} (\hat{n})
\int \dd[3]{y} T_{kl} \qty(t-r + \vec{y}\cdot \hat{n}, \vec{y}) 
\,.
\end{align}

This equation uses the stress-energy tensor as computed without considering the higher-order effects of gravity on matter (this is the meaning of ``linear'' in the linear wave equation \eqref{eq:wave-equation-einstein}); thus it is only a good approximation in the case of objects whose typical scale \(R\) is much larger, than, say, their Schwarzschild radius \(2GM\). 
The problem with this assumption is that it fails to hold precisely for the sources which we are most interested in since, as we shall see, they give out some of the most easily detectable gravitational radiation: binary compact objects near coalescence. 
Let us forget about this issue for now. 

For a gravitationally bound source like a binary, with total mass \(M\) and reduced mass \(\mu \), moving with speed \(v\), the virial theorem dictates that 
%
\begin{align}
\frac{1}{2} \mu v^2 = \frac{1}{2} G \frac{\mu M}{r} \implies v^2 = \frac{GM}{r} \sim \frac{R_s}{r} 
\,.
\end{align}

This means that the source not being very compact (\(R_s \ll r\)) is equivalent to it moving slowly (\(v \ll 1\)). 
Neither of these assumptions will hold in the end, but since we are using one we might as well use both. 

In Fourier space, this amounts to saying that the typical frequencies \(\omega \) for which the amplitude of the Fourier transform of the stress-energy tensor is large will satisfy \(\omega \vec{y} \cdot \hat{n} \lesssim \omega R \ll 1\). 
This means that we can expand the exponential in the Fourier transform; in the time domain this amounts to expanding in time around the retarded time \(t - r\): 
%
\begin{align}
T_{kl} (t-r + \vec{y} \cdot \hat{n}, \vec{y}) \approx T_{kl} (t-r, \vec{y}) + \vec{y} \cdot \hat{n} \partial_{t} T_{kl} + \frac{1}{2} (\vec{y} \cdot \hat{n})^2 \partial^2_{tt} T_{kl} + \order{(\vec{y} \cdot \hat{n})^3}
\,.
\end{align}

This allows us to write the resulting wave as 
%
\begin{align} \label{eq:gw-expansion-moments-stess-tensor}
h_{ij}^{TT} (t, \vec{x}) &= \frac{4G_N}{r} \Lambda_{ij, kl} \qty(S_{kl} + \sum_{L=1}^{\infty } \frac{1}{L!} \partial_{t}^{L} S_{kl | i_1 \dots i_L} n_{i_1} \dots n_{i_L})
\marginnote{To be computed at redarded time.}
\,,
\end{align}
%
where the moments of the stress tensor are defined as 
%
\begin{align}
S_{kl | i_1 \dots i_L} = \int \dd[3]{y} T_{kl} y_{i_1} \dots y_{i_L}
\,.
\end{align}

We can also analogously calculate moments for the energy density \(T_{00} \), which we denote \(M_{i_1 \dots i_L}\), and for the momentum density \(T_{0i}\), which we denote \(P_{k|i_1 \dots i_L}\). 

With integration by parts combined with the conservation of the (flat space!) stress-energy tensor \(\partial_{\mu } T^{\mu \nu }= 0\) we can relate the \(M\), \(P\) and \(S\) with equations such as \cite[eqs.\ 3.45--51]{maggioreGravitationalWavesVolume2007}
%
\begin{align}
\dot{M} &= 0  \\
\dot{P}_{i} &= 0 \\
\dot{M}_{ij} &= 2 P_{(i|j)}   \\
S_{ij} &= P_{i | j}
\,.
\end{align}

These all tell us something interesting: \(\dot{M} = 0\) and \(\dot{P}_{i} = 0\) are energy and momentum conservation, which seem to tell us that there is no energy nor momentum loss from \ac{GW} emission. 
This is an artifact, due to the assumptions of linear theory which neglect back-action on the source; fortunately we will still be able to compute the energy loss of the system.

The second useful fact is \(S_{ij} = \ddot{M}_{ij} / 2\): this allows us to write the lowest-order approximation of the expression of the wave from the source in terms of moments \eqref{eq:gw-expansion-moments-stess-tensor} as 
%
\begin{align}
h_{ij}^{TT}(t, \vec{x}) = \frac{2 G_N}{r} \Lambda_{ij, kl} \ddot{M}_{kl}
\,,
\end{align}
%
where 
%
\begin{align}
M_{kl}(t-r) = \int \dd[3]{y} T_{00}(t -r, \vec{y}) y_k y_l
\,.
\end{align}

since this expression only depends on the trace-free part of the moment \(M\) we can write it as a function of the traceless \textbf{quadrupole moment} 
%
\begin{align} \label{eq:quadrupole}
Q_{kl} = M_{kl} - \delta_{kl} M_{nn} / 3
= \int \dd[3]{y} \rho (t, \vec{y}) \qty(y^{i } y^{j} - \frac{1}{2} \delta^{ij} y^2)
\,.
\end{align}

This gives rise to the quadrupole formula 
%
\boxalign{
\begin{align} \label{eq:quadrupole-formula}
h_{ij}^{TT}(t, \vec{x}) = \frac{2 G_N}{r} \Lambda_{ij,kl} \ddot{Q}_{kl}(t-r)
\,.
\end{align}}

\subsubsection{Energy loss through gravitational radiation}

The problem of quantifying the energy carried by gravitational radiation is thorny: first of all, there is no universally valid way to split the perturbation from the background in the general case; also, at each point we can always apply the equivalence principle to recover flat spacetime up to first order. 
It is impossible to construct a \emph{local}, gauge invariant stress-energy tensor \(T_{\mu \nu }^{GW}\) for \ac{GW} radiation: the limit of the energy density contained in any volume will always vanish as that volume goes to zero. 

However, we \emph{can} define a tensor through an averaging procedure over many wavelengths and periods of the wave. There are different ways to do so, but a common one is the Landau-Lifshitz pseudotensor: 
%
\begin{align}
t_{\mu \nu } &= - \frac{1}{8 \pi G} \expval{R^{(2)}_{\mu \nu }- \frac{1}{2} \overline{g}_{\mu \nu } R^{(2)}}_{\text{mesoscopic}}  \\
&= \frac{1}{32 \pi G} \expval{\partial_{\mu } h_{\alpha \beta } \partial_{\nu }h^{\alpha \beta }}
\,,
\end{align}
%
where \(R^{(2)}_{\mu \nu }\) and \(R^{(2)}\) are the components of the Ricci tensor and scalar which are quadratic in the perturbation \(h_{\mu \nu }\); \(\overline{g}_{\mu \nu }\) is the background metric, and the averaging procedure is done on scales (wavelengths/periods) which are (much) larger than the typical wavelengths of the gravitational radiation considered, but (much) smaller than the typical wavelengths of the background \cite[sections 1.4.2, 1.4.3]{maggioreGravitationalWavesVolume2007}.

This pseudotensor can be used to describe the way in which, on large enough scales, the presence of \acsp{GW} does indeed curve spacetime.
Also, we can compute the energy flux passing through a surface a large distance from the source: if we use the quadrupole formula \eqref{eq:quadrupole-formula} for the gravitational perturbation, we can give an expression for the emitted power in terms of the third derivatives of the quadrupole as \cite[eq.\ 3.98]{maggioreGravitationalWavesVolume2007} 
%
\begin{align} \label{eq:energy-emission-quadrupole}
\dv{E}{t} = -\frac{G_N}{5} \expval{ \dot{\ddot{Q}}_{ij} \dot{\ddot{Q}}_{ij}}
\,.
\end{align}

Dimensionally, \(Q \sim \int \dd[3]{y} \rho r^2\) has units of \SI{}{kg m^2}; so \(\dot{\ddot{Q}}\) has units of \(\SI{}{kg m^2 / s^3} = \SI{}{W}\).

This means that the prefactor, which we wrote using \(c=1\), must really be \(G_N c^{n}/5\) with units of inverse power, which implies \(n = -5\).
Numerically, it is the inverse of \(5c^{5} / G_N \approx \SI{2e53}{W}\). 
This means that in order for \ac{GW} emission to be efficient we must have a large value for \(\dot{\ddot{Q}}\): let us estimate it in terms of the typical size of the system, \(R\), of its typical velocity \(v = \Omega R\) and of its mass \(M\). 
Each time derivative will roughly correspond to multiplication by a factor \(\Omega \), so \(\dot{\ddot{Q}} \sim \Omega^3 M R^2 = v^3 M / R\).

The power can then be estimated as 
%
\begin{align}
\dv{E}{t} \sim \frac{G_N}{5 c^{5}} v^{6} \frac{M^2}{R^2} 
= \underbrace{\frac{1}{5} \frac{c^{5}}{G_N}}_{\sim\SI{e52}{W}} \qty( \frac{v}{c})^{6} \qty( \frac{G_NM}{c^2 R})^2
\,,
\end{align}
%
which tells us that the most significant sources of \ac{GW} will be relativistic and compact.
This prediction has been validated in 2015 with the first detection of \ac{GW} from a \ac{BBH} system \cite{abbottObservationGravitationalWaves2016}, and again in 2017 with the first detection of \ac{GW} from a \ac{BNS} system \cite{abbottGW170817ObservationGravitational2017}. 

\subsection{Compact binaries}

We focus our attention towards a pair of inspiraling (compact) objects, which we will initially model as point masses.
The first thing we need to do is to write an expression the amplitudes in the two polarizations \(h_{+, \times }\) of the waves generated by a generic mass distribution with a mass moment \(M_{ij}\) \cite[eqs.\ 3.67--68]{maggioreGravitationalWavesVolume2007}:\footnote{We could also write these in terms of the traceless quadrupole moment \(Q_{ij}\); we follow the convention set forward by \textcite{maggioreGravitationalWavesVolume2007}.} 
%
\begin{align}
h_{+}(t, \hat{n}) &= \frac{G_N}{r} \qty(\ddot{M}_{11}' - \ddot{M}_{22}')\\
h_{+}(t, \hat{n}) &= \frac{G_N}{r} \qty(2\ddot{M}_{12}')
\,,
\end{align}
%
where \(M^{\prime }_{ij}\) are the components of the mass moment tensor in a frame whose \(z'\) axis is aligned to the observation direction.

This frame is not the natural one with which to describe a binary system: typically, we would want to align the \(z\) axis of the coordinates with the rotation axis. 
The rotation matrix between two systems depends on two angles, however one of these is more important than the other: aligning the \(z\) axes can be accomplished by a rotation of angle \(\iota \), while another rotation of angle \(\phi \) is needed in order to align the \(x\) and \(y\) axes as well. 
However, since (as we will discuss in a short while) the system is rotating in quasi-circular orbits, the second rotation only amounts to a time shift, or equivalently a phase in the oscillatory functions. 
For simplicity, we will neglect this phase freedom here and only include a variable phase at the end of the computations.

If the position of the bodies in the center-of-mass frame is 
%
\begin{align}
\vec{x}(t) = R \left[\begin{array}{c}
\cos(\Omega t) \\ 
\sin(\Omega t) \\ 
0
\end{array}\right]
\,,
\end{align}
%
then the mass moments read \(M_{ij} = \mu x_i (t) x_j(t)\) (where \(\mu = m_1 m_2 / (m_1 + m_2 ) \) is the reduced mass of the system), and going through the computation yields \cite[eq.\ 3.332]{maggioreGravitationalWavesVolume2007}
%
\begin{align} \label{eq:binary-waveform-1}
h_{+} (t) &= \underbrace{\frac{4 G_N \mu \Omega^2 R^2}{r}}_{A} \qty(\frac{1 + \cos^2 \iota }{2}) \cos( 2 \Omega t) 
\\
h_{ \times } (t) &= \underbrace{\frac{4 G_N \mu \Omega^2 R^2}{r}}_{A} \cos \iota  \sin( 2 \Omega t) 
\\
A &= \frac{4}{r} \qty(\frac{G\mathcal{M}_c}{c^2})^{5/3} \qty(\frac{\pi f _{\text{gw}}}{c})^{2/3} 
\,.
\end{align}

Approximating the motion of the bodies as circular during an orbit is called the \emph{adiabatic approximation}, since it assumes that no energy is lost in each orbit (as that would deform the circle). The variation of the orbital frequency \(\Omega \), which we shall compute shortly, must satisfy \(\dot{\Omega} \ll \Omega^2\). 
It is a rather good approximation for the early stages of the inspiral, as we shall see. 

The last expression we wrote for the amplitude of the emission reintroduces factors of \(c\), is expressed in terms of the emission frequency \(f _{\text{gw}} = \omega _{\text{gw}} / 2 \pi = \Omega / \pi \) and of the \emph{chirp mass}
%
\begin{align} \label{eq:chirp-mass}
\mathcal{M}_c = \frac{(m_1 m_2 )^{3/5}}{(m_1 + m_2 )^{1/5}} = \nu^{3/5} M 
\qquad \text{where} \qquad
\nu = \frac{\mu}{M}
\,.
\end{align}

\subsubsection{Energy evolution}

Writing the emission in this way allows us to give an expression for the emitted power 
%
\begin{align}
P = \frac{32}{5} \frac{c^{5}}{G} \qty(\frac{G \mathcal{M}_c \omega_{\text{gw}}}{2 c^3})^{10/3}
\,.
\end{align}

This allows us to describe the evolution of the orbit in this quasi-adiabatic context: the total energy of the binary system, by the virial theorem combined with Kepler's third law \(\Omega^2 R^3 = G M\), reads 
%
\begin{align}
E = - \frac{G m_1 m_2 }{2R}  = - \frac{G m_1 m_2 }{2} \frac{\Omega^{2/3}}{G^{1/3} M^{1/3}} = - \qty( \frac{G^2 \mathcal{M}_c^{5} \omega^2 _{\text{gw}}}{32} )^{1/3}
\,.
\end{align}

With these two expressions at hand, we can impose \(P = -  \dot{E}\), which we can write in the form \(\dot{\omega} _{\text{gw}} \propto \omega^{n}\) for some \(n\): since \(\dot{E}\propto \omega _{\text{gw}}^{-1/3} \dot{\omega} _{\text{gw}}\), \(n\) will be \(11/3\), and the specific expression will look like 
%
\begin{align}
\dot{\omega} _{\text{gw}} = \frac{12}{5} \sqrt[3]{2} \qty( \frac{G \mathcal{M}_c}{c^3})^{5/3} \omega _{\text{gw}}^{11/3}
\,.
\end{align}

This equation is in the form \(\omega^{-11/3} \dd{\omega } = K \dd{t}\); integrating from point 0 to point 1 we find \(3K(t_1 - t_0 ) / 8 = \omega_0^{-8/3} - \omega_1^{-8/3}\); if we treat this as an initial value problem we can see that as \(t_1 \) increases there is only a finite ``budget'' on the right-hand side: when \(3K (t_1 - t_0 ) / 8 = \omega_0^{-8/3}\), \(\omega_1^{-8/3}\) must go to 0 so \(\omega_1 \) must diverge. 
The divergence itself is unphysical since, besides corresponding to a breakdown of some of our approximations, it comes about when considering point masses, while compact objects have a finite size and at some point collide. 

Nevertheless, the moment of divergence happens close enough to the actual merger of the objects, therefore it is useful to call it \(t_c\) and to define a time coordinate as \(\tau = t_c - t\).
In terms of this, the solution to the equation reads 
%
\begin{align}
f _{\text{gw}} (t) = \frac{\omega _{\text{gw}} (t)}{2 \pi } = \frac{1}{2 \pi }\qty( \frac{ 3 K \tau }{8})^{-3/8} = \frac{1}{\pi } \qty(\frac{ 5}{256 \tau })^{3/8} \qty(\frac{G \mathcal{M}_c}{c^3})^{-5/8}
\,.
\end{align}

At this point we can notice an interesting scaling property: if we map \(f _{\text{gw}} \to \alpha f _{\text{gw}}\), \(\tau \to \tau / \alpha \) and \(\mathcal{M}_c \to \mathcal{M}_c / \alpha \) (or, specifically, we divide each mass by \(\alpha \)) the equation still holds! 
This result turns out to hold in general, not only at the linear, quadrupole level. 
[FIND CITATION FOR THIS STATEMENT]

This expression for \(f _{\text{gw}}\) can be substituted into the gravitational waveforms \eqref{eq:binary-waveform-1}; we must also update the phase term, since \(2 \Omega t\) is the phase for a uniformly circular orbit: we will want to use integrate the angular velocity, to find 
%
\begin{align} \label{eq:phase-quadrupole-emission}
\Phi (t) &= \int^{t} \omega _{\text{gw}}(\widetilde{t}) \dd{\widetilde{t}} = 2 \int^{t} \Omega (\widetilde{t}) \dd{\widetilde{t}}  \\
&= -2 \qty(\frac{5 G \mathcal{M}_c}{c^3})^{-5/8} \tau^{-5/8} + \const
\,.
\end{align}

In terms of this, the amplitude in the two polarizations will read \cite[eqs.\ 4.31--32]{maggioreGravitationalWavesVolume2007}
%
\begin{align} \label{eq:amplitude-quadrupole-emission}
h_{+} (t) &= \frac{1}{r} \qty(\frac{G \mathcal{M}_c}{c^2})^{5/4} \qty( \frac{5}{c \tau })^{1/4} \qty(\frac{1 + \cos^2\iota }{2}) \cos \Phi (t) \\
h_{ \times } (t) &= \frac{1}{r} \qty(\frac{G \mathcal{M}_c}{c^2})^{5/4} \qty( \frac{5}{c \tau })^{1/4} \cos \iota  \sin \Phi (t) 
\,.
\end{align}

As we will discuss, it is convenient to have expressions for the Fourier transforms of these. An analytic computation of the integrals is intractable, but we can make use of a technique known as \ac{SPA}, which is discussed in detail in appendix \ref{sec:spa}. 
The final expressions for the Fourier-domain waveforms are equations \eqref{eq:spa-waveforms}. 

\subsubsection{Parameters for a \ac{CBC} waveform}

The waveform we wrote explicitly depends on: 
\begin{enumerate}
    \item the chirp mass \(\mathcal{M}_c\), defined in \eqref{eq:chirp-mass};
    \item the distance \(r\), which in a more general cosmological context should be replaced by the \emph{luminosity distance} \(D_L\);
    \item the inclination \(\iota \), which is the angle between the observation direction and the total angular momentum of the system.
\end{enumerate}

These are only some of the parameters which must be considered when discussing a \ac{CBC}. We will now discuss the full set of parameters which can be used to fully describe the binary system generating the waveform \cite[eq.\ 21]{breschiTtBajesBayesian2021}. 

\paragraph{Arrival time and initial phase}

The time at which any given waveform arrives at Earth is arbitrary, as is the global phase of the waveform. 
In practice, one will typically analyze the output from a detector in batches, and in each of these the analysis will be performed in Fourier space. 
Therefore, it is relevant to see how the Fourier-domain waveform responds to a time and phase shift: what is the transform of \(f(t-t_0 ) e^{i \varphi_0} \)? It comes out to be 
%
\begin{align}
\int_{\mathbb{R}} f(t-t_0 ) e^{i \varphi_0 } e^{i \omega t} \dd{t}  = e^{i \omega t_0 + i \varphi_0} \underbrace{\int_{\mathbb{R}} f(t-t_0 ) e^{i \omega (t-t_0 )} \dd{(t-t_0 )}}_{\widetilde{f}(\omega )}
\,,
\end{align}
%
so a time shift corresponds to the addition of a linear term to the phase, while the phase can be directly moved from the time to the frequency domain. 

\paragraph{Sky position and polarization}

The wave will be coming from a sky position, which we can describe with respect to a given coordinate system through two angles; it is convenient to use right ascension \(\alpha \) and declination \(\delta \). 
Also, the polarization of the gravitational waves can have an arbitrary angle with respect to the observation direction: we denote this angle as \(\psi \). 
% One might wonder whether this parameter will be fully degenerate with the initial phase \(\phi_0\) and thus unmeasurable, but it can in fact be measured if we have a network of detectors whose response functions to either polarization are different. 

In more technical terms, a general waveform arriving our detector will be a superposition of \acsp{GW} in the TT gauge \eqref{eq:TT-gauge-gw} with varying \(\alpha \), \(\delta \), \(\psi \) as well as different frequencies; we can write it as \cite[eq.\ 1.58]{maggioreGravitationalWavesVolume2007}
%
\begin{align} \label{eq:generic-gw}
h_{ab} (t, \vec{x}) &= \sum _{\text{pol} = +, \times }
\int_{- \infty  }^{\infty } \dd{f} \int \dd[2]{\hat{n}} \widetilde{h}_{\text{pol}} (f, \hat{n}) e^{\text{pol}}_{ab}(\hat{n}, \psi ) e^{-2 \pi i f (t - \hat{n}\cdot \vec{x} / c )} 
\,,
\end{align}
%
where \(\hat{n} = \hat{n} (\alpha , \delta )\) is the vector describing the propagation direction of a specific component --- in the context of a \ac{CBC} the distribution of the \(e^{\text{pol}}_{ab}(\hat{n}, \psi )\) will include a Dirac delta centered on the position of the source in the sky, this more general expression can be useful, for example, in the context of \ac{SGWB}. 

The tensors \(e_{ab}\) are basis tensors: in a frame where \(\hat{n}\) and \(\psi \) are chosen such that the \(+\) polarization stretches the \(x\) and \(y\) axes, they read 
%
\begin{align}
e_{ij}^{+} (\hat{n}=\hat{z}, \psi =0) &= u_i u_j - v_i v_j = \left[\begin{array}{ccc}
1 & 0 & 0 \\ 
0 & -1 & 0 \\ 
0 & 0 & 0
\end{array}\right]  \\
e_{ij}^{ \times } (\hat{n}=\hat{z}, \psi = 0) &= u_i v_j + v_i u_j = \left[\begin{array}{ccc}
0 & 1 & 0 \\ 
1 & 0 & 0 \\ 
0 & 0 & 0
\end{array}\right]
\,,
\end{align}
%
where \(u\) and \(v\) are two unit vectors defining the polarization direction; they are always chosen to be orthogonal, so fixing the angle \(\psi \) is enough to determine them both once we give \(\hat{n}\). 

The aforementioned expansion glosses over a technical issue: the three-dimensional Fourier transform of the \ac{TT} gauge waveform is in the form \(h_{ij}(x) \sim \int \dd[3]{k} A_{ij}(k) e^{ikx}\), so in order to write it like we did we need to define 
%
\begin{align}
\frac{f^2}{c^3} A_{ij} (f, \hat{n}) = \sum _{\text{pol}= +, \times }
\widetilde{h}_{\text{pol}} (f, \hat{n}) e^{\text{pol}}_{ij}(\hat{n})
\,,
\end{align}
%
where we reparametrized the wavevector \(\vec{k}\) through frequency \(f\) and direction \(\hat{n}\), and exploited the decomposition \(\dd[3]{k} = k^2 \dd{k} \dd[2]{\hat{n}}\).

\paragraph{Masses and mass ratio}

The waveform to the lowest order we described only depends on the chirp mass \(\mathcal{M}_c\), but as we will see later at higher order there appears a dependence on the ratio of the two masses: \(q = m_1 / m_2 \), where \(m_i\) are the masses of the two compact objects. 
It is customary to fix one of the two masses to always be the largest, therefore constraining \(q \lesseqgtr 1\); we choose the convention \(q \geq 1\).

A possible parametrization for the two masses is \((\mathcal{M}_c, q)\), another is \((M, q)\): these are equivalent, but in the following work we will typically use the latter because of the aforementioned scaling property of the waveforms: we conventionally work with respect to the dimensionless frequency \(Mf\) and the dimensionless time \(t / M\) in order to have one less parameter. 

\paragraph{Spin}

Compact objects can spin, and the spin of each of the two is a vector with three independent components. 
These vectors \(\vec{S}_i\) have units of angular momentum, but they are typically rescaled as 
%
\begin{align}
\vec{\chi}_i = \frac{c\vec{S}_i}{G m_i} \in [-1, 1] 
\ \text{in magnitude}
\,.
\end{align}

The constraint of \(\abs{\chi_i} < 1\) can be approached by realistic models of \acsp{BH}, while realistic models of \acsp{NS} typically are constrained to spin much less, at the most \(\abs{\chi _i} \lesssim 0.7\) \cite{loSPINPARAMETERUNIFORMLY2011}.
This constraint is hard to experimentally verify since one is faced with the degeneracy between the components \(\chi_{z,i}\) aligned with the direction of the angular momentum and the mass ratio \(q\); in the analysis of the merger GW170807 two sets of prior distributions for the spins were used, \(\abs{\chi } < 0.05\) and \(\abs{\chi } < 0.89\), for this reason \cite{abbottGW170817ObservationGravitational2017}. We have theoretical reasons to believe that the low-spins priors might correspond to a more physically meaningful scenario, but the degeneracy means that the high-spin case is not experimentally excluded, so we must still consider it a possibility.

\paragraph{Tidal polarizability of neutron stars}

If our compact objects are not black holes, they might be able to deform. The astrophysically motivated scenario we think of is that of neutron stars, but the following characterization can also apply to any extended compact objects. For concreteness' sake, we will refer to \acsp{NS}. 

This is a complicated process, but we can try to capture its most significant part by limiting ourselves to the quadrupole order of the deformation of the star. 
The discussion of this effect is clearest in the Newtonian context, but it can be generalized to \ac{GR} \cite[section 14.4.1]{maggioreGravitationalWavesVolume2018}.

Tidal effects are described by the tidal tensor \(\mathcal{E}_{ij} = - U_{, ij}\), the traceless\footnote{This tidal tensor describes the effect of one star's gravitational field on the other, so its source is not where we compute it, which is a convoluted way of saying that it should be taken to be a solution of \(\nabla^2 U = 4 \pi \rho \) in vacuum: therefore, \(\nabla^2 U = - \Tr[\mathcal{E}_{ij}] = 0\).} Hessian of the Newtonian potential \(U\). In the weak-field relativistic case, this corresponds to part of the Riemann tensor: \(\mathcal{E}_{ij} = c^2R_{0i0j}\). 

We can describe the effects on the deformed star of such a tidal stress by looking at its quadrupole moment \(Q_{ij}\) \eqref{eq:quadrupole}.
To linear order and neglecting any time dependence, we will have the relation 
%
\begin{align}
\mathcal{E}_{ij} = - \lambda Q_{ij}
\,.
\end{align}

\todo[inline]{find better explanation for minus sign here and in def of tidal tensor}

This parameter \(\lambda \) describes the deformability of the star; if it is smaller the star deforms less in response to tidal perturbations. 
In practice the parameter used is not \(\lambda \) but two ways to rescale it: the \(l=2\) (quadrupole) \textbf{Love number} 
%
\begin{align}
k_2 = \frac{3}{2} \frac{G \lambda }{R^{5}}
\,,
\end{align}
%
where \(R\) is the radius of the deformed \ac{NS}, or the \textbf{tidal deformability} 
%
\begin{align}
\Lambda = \frac{2}{3} k_2 \qty(\frac{Rc^2}{Gm})^{5} 
\,,
\end{align}
%
where \(m\) is the mass of the deformed \ac{NS}. 
The value of these parameters depends on the specific equation of state used in the model; \(k_2\) is typically of the order of \(10^{-1}\), and in the expression for \(\Lambda \) we have a fifth power of the \emph{compactness} \(\sigma = R c^2 / Gm\) of the \ac{NS} --- this is expected to be a small number, but larger than 3,\footnote{A limit on the maximum redshift of radiation from the surface of a \ac{NS} was calculated by \textcite{lindblomLimitsGravitationalRedshift1984} under the assumptions of causality and stability for the nuclear matter, which is equivalent to a compactness limit of \(\sigma \geq 2.83\) \cite{lattimerNeutronStarObservations2007}. } so its fifth power will be of order \(10^{3 \divisionsymbol 4}\). This means that the deformability \(\Lambda \) will typically be of the order \(10^{2 \divisionsymbol 3}\). 

The Love number is useful since it is an expansion parameter for the gravitational field of the deformed \ac{NS}. This field has contributions from both its quadrupole moment, which is due to the tidal deformation from the other star, and from the gravitational field of the other star which we can expand in terms of the tidal field around the center of the deformed star: \(U _{\text{ext}} \approx - \mathcal{E}_{ij} x_{i} x_{j} / 2\). 

The gravitational field of the deformed star also changes under the effect of the deformation: including the first term in its expansion, which depends on the quadrupole moment, the potential reads  
%
\begin{align}
U _{\text{int}} \approx \frac{Gm}{r} + \frac{3G}{2r^3} \frac{x_i}{r} \frac{x_j}{r} Q_{ij}
\,,
\end{align}
%
where \(r = \abs{x}\) is the radius from the center of the deformed star.

We can then see that under the assumption of linear dependence of the quadrupole on the tidal tensor the two potentials can be added (since we are still in the realm of linear theory):\footnote{A technical fact to remember here is that the expansions do not naturally match: the expansion of the internal potential is in orders of \(1/r\), while the expansion of the external one is in orders of \(r\). } 
%
\begin{align}
U \approx \frac{Gm}{r} - \frac{1}{2} \mathcal{E}_{ij} x_i x_j \qty[1 + 2 k_2 (R / r)^{5}]
\,.
\end{align}

The meaning of \(\Lambda \) is less immediate ---  the first correction to the phasing of the waveform which appears is\footnote{The meaning of the \ac{PN} expansion and of the expansion in orders of \(v/c\) will be discussed in [SEC]. } \cites[eq.\ 14.231]{maggioreGravitationalWavesVolume2018}{flanaganConstrainingNeutronStar2008}
%
\begin{align}
\Delta \Psi^{\text{tidal}}_{\text{5PN}} = - \frac{117}{256} \frac{m^2}{m_1 m_2 } \widetilde{\Lambda } \qty(\frac{v}{c})^{5} 
\qquad \text{where} \qquad
\widetilde{\Lambda} = \frac{16}{13} \frac{(m_1 + 12m_2 ) m_1^{4} \Lambda_1 + (m_2 + 12 m_1 ) m_2^{4} \Lambda_2 }{(m_1 +m_2 )^{5}}
\,.
\end{align}

The parameter \(\widetilde{\Lambda}\), which depends on the tidal deformabilities of the two stars \(\Lambda_{1, 2}\), is called the \emph{reduced tidal parameter}. 

One would expect to see \(\Lambda = 0\) for black holes, and this is true in the nonspinning case; on the other hand, Kerr black holes can exhibit small but nonvanishing deformability. \textcite{letiecSpinningBlackHoles2021} recently showed that, for example, \acsp{BH} with dimensionless spins of the order \(\chi \sim 0.1\) can exhibit Love numbers on the order of \(k_\ell \sim \num{2e-3}\), around two orders of magnitude less than \ac{NS} values. 
The effect this will have on \ac{GW} emission is yet to be determined, but one can suspect it will be small.

\paragraph{Eccentricity}

A general binary system will have a nonvanishing eccentricity \(e \in [0, 1)\), such that the two semiaxes of the ellipse read \cite[eqs.\ 4.51]{maggioreGravitationalWavesVolume2007}
%
\begin{align}
a = \frac{R}{1- e^2} \qquad \text{and} \qquad
b = \frac{R}{\sqrt{1 - e^2}}
\,.
\end{align}

It can be shown that, for a Keplerian orbit, the semimajor axis \(a\) and the eccentricity \(e\) depend on the total energy \(E\) and angular momentum \(L\) as \cite[eqs.\ 4.50, 4.53]{maggioreGravitationalWavesVolume2007}
%
\begin{align}
e = \sqrt{1 + \frac{2 EL^2}{G^2 m^2 \mu^3}} 
\qquad \text{and} \qquad
a = \frac{Gm \mu }{2 \abs{E}}
\,.
\end{align}

In order to fully describe the \ac{GW} emission from a binary system we must then also account for eccentricity; this parameter can also change as time progresses, and in order to describe this process we need a second evolution equation.
This can be found thanks to the quadrupole angular momentum emission formula, which is analogous to the energy emission one \eqref{eq:energy-emission-quadrupole} \cite[eq.\ 3.99]{maggioreGravitationalWavesVolume2007}:\footnote{The angular momentum emitted in this expression is considered to be both spin and orbital angular momentum.} 
%
\begin{align}
\dv{L^{i}}{t} = - \frac{2G_n}{5c^{5}} \epsilon^{ikl} \expval{ \ddot{M}_{ka} \dot{\ddot{M}}_{la}} 
\,.
\end{align}

This allows us to make a system of two equations, \(\dv*{E}{t}\) and \(\dv*{L}{t}\), which we can reparametrize as equations for \(\dot{a}\) and \(\dot{e}\) \cite[eqs.\ 4.116--17]{maggioreGravitationalWavesVolume2007}: 
%
\begin{align}
\dot{a} &= - \frac{64}{5} \frac{G^3 \mu m^2}{c^{5} a^3}
\frac{1}{(1-e^2)^{7/2}} \qty(1 + \frac{73}{24} e^2 + \frac{37}{96}e^{4})  \\
\dot{e} &= - \frac{304}{15}  \frac{G^3 \mu m^2}{c^{5} a^4} 
\frac{e}{(1-e^2)^{5/2}} \qty(1 + \frac{121}{304} e^2)
\,,
\end{align}
%
which can be (nontrivially) analytically solved, yielding the result shown in figure \ref{fig:eccentricity}. 

\begin{figure}[ht]
\centering
\includegraphics[width=\textwidth]{figures/eccentricity}
\caption{We show the behaviour of the analytic solution for the semimajor axis \(a(e)\) \cite[eq.\ 4.126]{maggioreGravitationalWavesVolume2007}; the \(x\) axis is on a logit scale, meaning that it is proportional to \(\log (e / (1-e))\). The quantity on the \(y\) axis is not \(a(e)\), but \(a\) is proportional to it: ratios in the form \(a(e_1) / a(e_2)\) are correctly represented in this graph.}
\label{fig:eccentricity}
\end{figure}

It can be clearly seen that eccentricity decreases as the semimajor axis decreases.
Typical astrophysical binaries need to shrink by several orders of magnitude before merging --- the notorious pulsar in a binary system detected by \textcite{hulseDiscoveryPulsarBinary1975} has a semimajor axis of \(a \approx \SI{2e9}{m}\) and an eccentricity of \(e \approx 0.617\), while the frequency of the \acsp{GW} it emits is \(f _{\text{gw}} = 2 / P \approx \SI{7e-5}{Hz}\) \cite{taylorNewTestGeneral1982}. 

If we were to wait until this reached a frequency range where next generation ground-based detectors might hope to detect it, \(f \sim \SI{10}{Hz}\), we would expect to see (using Kepler's third law \(a^3 f^2 = \const\)) a reduction of semimajor axis by a factor \(\sim 2700\): it would go off-scale in figure \ref{fig:eccentricity}, so we would expect to basically have \(e \approx 0\). 
One can then see that this holds for a very wide range of possible values of \(e\). 

This is the reason why eccentricity is often not considered in \ac{GW} modelling, while it is of crucial importance in astrophysical population studies, since it has a large effect on the emitted power (enhancing it) in the early stages of the inspiral.

Eccentric binaries are by no means excluded by this line of argument; there are astrophysical contexts in which they might be generated with high eccentricities combined with already small radii, which would allow \(e > 0\) to be detectable even in the \ac{LIGO}-Virgo band.
Eccentric binary \ac{GW} models do exist \cite{favataConstrainingOrbitalEccentricity2021}, and ignoring eccentricities may lead to bias in parameter estimation.\footnote{It is interesting to note that these models reach into ``relativistic territory'', and therefore must also include effects such as periastron precession --- a good resource on these, based on \cite{favataGravitationalwaveMemoryEccentric2011}, is the \href{Sounds of Spacetime}{https://www.soundsofspacetime.org/elliptical-binaries.html} website, which also provides audio renditions of the waveforms.}

\paragraph{Summary}

The parameter vector \(\vec{\theta}\) describing a binary sysem can be divided into internal and external parameters as 
%
\begin{align}
\vec{\theta} = (\underbrace{M, q, \vec{\chi}_1, \vec{\chi}_2, \Lambda_1, \Lambda_2,}_{\theta _{\text{int}}} \underbrace{D_L, \iota, \alpha, \delta, \psi, t_0, \phi_0 }_{\theta _{\text{ext}}})
\,,
\end{align}
%
where the dependence of the waveform on the external parameters is well-understood and easy to analytically calculate, while the dependence on the internal parameters (barring \(M\)) is complicated.

\subsection{Interferometers and data analysis}

Having seen how a gravitational waveform from a \ac{CBC} might look to lowest order, we move to a discussion of the detection of these waveforms with interferometric techniques. 

The response of any detector to a \ac{GW} is a\footnote{Current interferometric detectors have a single scalar response because of their Michelson-Morley design with arms at \SI{90}{\degree}; planned detector such as the Einstein Telescope will exhibit multiple scalar inputs \cite[section 5.3.2]{etscienceteamEinsteinGravitationalWave2011}. The following analysis still applies, each scalar input can be treated analogously; having multiple (from one multi-output detector or from a network) is incredibly beneficial for the accuracy of measurements.} scalar output, which will be in the form 
%
\begin{align}
h(t) = D_{ij} h_{ij}
\,,
\end{align}
%
where \(D_{ij}\) is known as the \emph{detector tensor}. 
We want to apply this expression to the generic one for a gravitational wave \eqref{eq:generic-gw}, which we can simplify by
\begin{enumerate}
    \item removing the integral in \(\dd[2]{\hat{n}}\): in the case of a \ac{CBC} this is an excellent approximation;
    \item removing the dependence on \(\vec{x}\): this is called the \emph{short-arm} approximation, which is warranted by the fact that our detectors have arms with lengths \(L \sim \SI{3}{km}\) and are most sensitive for \ac{GW} frequencies of \(f \sim \SI{100}{Hz}\), while the frequencies corresponding to \(L\) are \(f \sim
     \SI{100}{kHz}\).\footnote{It is actually possible to not use this approximation, and in fact it is advisable not to: modern interferometers use power recycling techniques, which allow for the effective length of the arms to be much longer than their physical one. In fact, the optimal detection strategy for any given \ac{GW} frequency is to have a detector whose arms are a quarter of the \ac{GW} wavelength long \cite[eq.\ 9.33]{maggioreGravitationalWavesVolume2007} --- this balances the effect of the deformation due to the \ac{GW} changing sign during the time of flight of any specific photon with the ``stacking'' effect of the photon taking a longer path through the deformed space.}
\end{enumerate}

This leads to the following expression for the observed signal:
%
\begin{align}
h(t) &= \sum _{\text{pol}= +, \times } \underbrace{D^{ij} e_{ij}(\hat{n}, \psi )}_{F _{\text{pol}}}  \underbrace{\int_{- \infty }^{\infty } \dd{f} \widetilde{h} _{\text{pol}} e^{-2 \pi i f t}}_{h _{\text{pol}}(t)}  \\
&= F_+ h_+ (t) + F_\times h_\times (t)
\,.
\end{align}

The \emph{detector pattern functions} \(F_+\) can be computed explicitly; for a Michelson-Morley interferometer they read 
%
\begin{align}
F_{+} &= \frac{1}{2} \qty(1 + \cos^2 \theta ) \cos 2 \phi \cos 2 \psi - \cos \theta \sin 2 \phi \sin 2 \psi   \\
F_{+} &= \frac{1}{2} \qty(1 + \cos^2 \theta ) \cos 2 \phi \sin 2 \psi + \cos \theta \sin 2 \phi \cos 2 \psi  
\,,
\end{align}
%
where \(\theta \), \(\phi \) are the two angles describing the direction the \ac{GW} is coming from in a frame aligned with the axes of the detector --- they will depend on the sky position of the source (\(\alpha \), \(\delta \)) as well as the orientation of the detector in space (which depends on well-known parameters such as its latitude, the orientation of the earth at each time and so on).

\subsubsection{Matched filtering}

We have seen how the \ac{GW} signal will look to our detector: \(h(t)\), but in practical experiments what we will measure will be in the form \(s(t) = h(t) + n(t)\),\footnote{The ``noise'' described here is not actually what is measured: the output of the detector is not \(h(t)\) but it is a linear function of it --- even without accounting for the technical details of the measurement, the quantity measured is the intensity of the light at the dark fringe of the detector, not directly \(h\).} and typically the magnitude of the noise timeseries \(n(t)\) will be much larger than the magnitude of the signal. 

This poses an issue both for the detection of a signal and for the analysis of a signal which has been identified as such. 
The technique we will describe here, matched filtering, has applications in both branches of \ac{GW} data analysis.

The idea is to define a \emph{filter}, a linear map from the signal timeseries to \(\mathbb{R}\), in such a way that its value is low if there is no signal, and it is high if there is a signal of a certain shape. 
In general, such a function can be expressed as 
%
\begin{align}
s(t) \to \hat{s} = \int \dd{t} s(t) K(t)
\,
\end{align}
%
for some filter function \(K(t)\), which we can select arbitrarily.
How can we determine the best choice of \(K\)?
We want to maximize the \emph{distinguishability} between true signals and random noise, which we can quantify through the \ac{SNR}:\footnote{Properly speaking, this quantifies the distinguishability only under the assumption of zero-mean noise, otherwise we could make it arbitrarily large by adding a constant to \(s(t)\).}
%
\begin{align}
\text{SNR} = \frac{\mathbb{E}(\hat{s} | \text{presence of \(h\)}) }{\sqrt{\var{\hat{s} | \text{absence of \(h\)}}}} = \frac{S}{N}
\,,
\end{align}
%
where we compute the root of a variance for \(N\) since if there is only noise we expect \(\hat{s} = \int \dd{t} n(t) K(t)\) to be a random variable.

In order to properly express this, let us discuss our assumptions about the statistical properties of the noise: the simplest noise we can characterize is 
\begin{enumerate}
    \item stationary: its statistical properties are unchanging in time. This is not true in real detectors, but if the variation is slow enough one can work with ``local'' properties, on the scale of hours or days.
    \item zero-mean: \(\expval{n(t)} =0\).
    \item uncorrelated in Fourier space: this can be stated simultaneously with the definition of the variance of each Fourier component, which is expressed through the single sided\footnote{The distinction between the single- and double-sided \ac{PSD} depends on whether we want to use negative frequencies in the integral to recover the variance at each time or not: 
    %
    \begin{align}
    \expval{n^2(t)} 
    = \int_{0}^{\infty } \dd{f} S_n^{\text{single-sided}}(f)
    = \int_{-\infty }^{\infty } \dd{f} S_n^{\text{double-sided}}(f)
    \,.
    \end{align}
    
    Since the noise is real-valued, these two are simply related by \(S_n^{\text{single-sided}} = S_n^{\text{double-sided}}/2\).
    } \textbf{\ac{PSD}} \(\expval{\widetilde{n}^{*}(f ) n(f )} = \delta (f - f' ) S_n(f ) / 2\). 
    \item Gaussian: each Fourier component is normally distributed around zero, with a variance described by the \ac{PSD}.
\end{enumerate}

The \ac{PSD}, as defined, has the dimension of an inverse frequency; since it describes a variance it is often useful to discuss its square root, the \emph{spectral strain sensitivity}, or \emph{amplitude spectral density} \(\sqrt{S_n}\), with dimensions \(1/ \sqrt{ \SI{}{Hz}}\). 

With these assumptions, we can write the \ac{SNR}, moving to Fourier space, as \cite[eq.\ 7.45]{maggioreGravitationalWavesVolume2007}
%
\begin{align} \label{eq:snr}
\frac{S}{N} = \frac{\int_{\mathbb{R}} \dd{f} \widetilde{h}(f) \widetilde{K}^{*}(f) }{\sqrt{\int_{\mathbb{R}} \dd{f} (S_n(f) / 2) \abs{\widetilde{K}(f)}^2}} = \frac{(u|h)}{\sqrt{(u|u)}}
\,,
\end{align}
%
where we defined the \textbf{Wiener product} between two real-valued signals \(a\) and \(b\) as the Fourier-space expression \cites{finnDetectionMeasurementGravitational1992}[eq.\ 7.46]{maggioreGravitationalWavesVolume2007}
%
\begin{align}
(a | b) = 4 \Re \int_{0}^{\infty } \dd{f} \frac{ \widetilde{a}^{*} (f) b(f)}{S_n(f)}
\,.
\end{align}

The reason for the presence of the real part is that we want this to math the previous expression, where we know the signal \(S\) to be real-valued; the factor 4 is a combination of the factor of \(2\) in the definition of the \ac{PSD} and the fact that we restrict the integral to positive frequencies only, using the fact that the negative-frequency part gives the same contribution for real-valued signals.

We also defined the signal \(u\), which is defined so that its Fourier transform reads \(\widetilde{u}(f) = \widetilde{K}(f) S_n(f) / 2\), which allows the expression with the Wiener product to match the previous one.

This expression can then be written as \(S/N = (\hat{u} | h)\), where \(\hat{u} = u / \sqrt{(u|u)}\) This is then maximized by \(\hat{u}\) parallel to \(h\) with respect to the metric defined by the Wiener product: \(\hat{u} \propto h\) means that 
%
\begin{align}
\widetilde{K}(f) \propto \frac{\widetilde{h}(f)}{S_n(f)}
\,.
\end{align}

In other words, the best way to find a signal buried in noise is to scale the Fourier-domain expression for the filter by the amplitude of the noise. 

\paragraph{Whitening}

An alternative way to write the same expression is through the concept of \emph{whitening}: if \(S_n(f)\) is known, we can transform any signal \(a(t)\) into 
%
\begin{align}
a_w(t) \qquad \text{such that} \qquad \widetilde{a}_w (f) = \frac{\widetilde{a}(f)}{\sqrt{S_n(f) / 2}}
\,.
\end{align}

In other words, we are mapping a signal into another signal where all the noise Fourier components are uniformly scaled: white noise. 

In terms of the whitened signals, the Wiener product just reads 
%
\begin{align}
(a|b) = (a_w | b_w)_w = 2 \Re \int_{0}^{\infty } \dd{f} \widetilde{a}_w^{*} (f) \widetilde{b}_w(f) = \int_{- \infty }^{\infty } \dd{t} a_w(t) b_w(t)
\,,
\end{align}
%
so it can also be computed in the time domain.

\paragraph{Optimal \ac{SNR}}

The expression for the \ac{SNR}, if we are using the optimal filter, is then (in terms of an arbitrary constant \(C\)):
%
\begin{align}
\text{optimal \ac{SNR}} = \frac{(Ch | h)}{\sqrt{(Ch|Ch)}} = \sqrt{(h|h)} 
= 4 \int_{0}^{\infty } \dd{f} \frac{\abs{\widetilde{h}(f)}^2}{S_n(f)}
\,.
\end{align}

Amplitude strain profiles, as well as Fourier transforms of signals, are often plotted with log-scales: in order to have an intuition for this quantity we can reframe it as 
%
\begin{align}
\text{optimal \ac{SNR}} &=
\int_{- \infty }^{\infty } \dd{\log f}  \frac{h_c^2 (f)}{h_n^2(f)}  
\marginnote{Used \(\dd{\log f} = \dd{f} / f\).}\\
h_c (f) &= 2 f \abs{\widetilde{h}(f)}  \\
h_n (f) &= \sqrt{f S_n(f)}
\,,
\end{align}
%
where the quantities \(h_c\) and \(h_n\) are called the \textbf{characteristic strains} of signal and noise \cite[eqs.\ 17--19]{mooreGravitationalwaveSensitivityCurves2015}. 
They are both dimensionless, the way this integral is expressed allows us to integrate ``by eye'': if we plot log-characteristic strain against log-frequency, the positive area between \(h_c\) and \(h_n\) will be proportional to the optimal \ac{SNR}. 

Alternatively, we can define an ``amplitude signal strain spectral density'': 
%
\begin{align}
\sqrt{S_h (f)} = 2 \sqrt{f} \abs{\widetilde{h}(f)}
\,,
\end{align}
%
which is comparable to \(\sqrt{S_n(f)}\), in the sense that it also has units of \SI{}{Hz^{-1/2}}.

\subsubsection{Parameter inference}

Modern data analysis techniques for \ac{GW} signals are Bayesian by necessity: we need to extract estimates for the parameters generating a signal of which we only have one measurement. 

The quantity we want to extract from our analysis is the \textbf{posterior probability density function} \(\mathbb{P}(\vec{\theta} | s)\), where \(\vec{\theta}\) is the parameter vector while \(s\) represents the data from the detector. 
If we integrate it in a certain hyper-volume \(\Omega \), we get \(\int_{\Omega } \dd[n]{\vec{\theta}} \mathbb{P}(\vec{\theta} | s)\): the answer to the question ``given the data we measured, what is the probability that the parameters of the system were contained in the region \(\Omega \)?''

The way we write these probabilities is a compactification of notation: 
after the ``given'' symbol we should also always ideally include all the other assumptions about the signal, the noise, and the way they are combined into \(s = h + n\). We will always leave this implicit, but we understand that the probabilities calculated will always be model-dependent. 

A Bayesian approach starts by applying Bayes' theorem: 
%
\begin{align}
\underbrace{\mathbb{P}(\vec{\theta}| s )}_{\text{posterior}} = \frac{\mathbb{P}(s | \vec{\theta}) \mathbb{P}(\vec{\theta})}{\mathbb{P}(s)}
\propto \underbrace{\mathbb{P}(s | \vec{\theta})}_{\text{likelihood}} \underbrace{\mathbb{P}(\vec{\theta})}_{\text{prior}}
\,,
\end{align}
%
where the reason we can neglect the dependence on \(\mathbb{P}(s)\) is that it is a constant: since probability density functions are normalized to have unit integral over the whole space, we can worry about normalization in the end. 
Also, there are techniques to explore the posterior probability density space (such as \ac{MH} sampling) which can use an unnormalized version of the posterior.

The prior, which encodes our prior belief about the potential values of the parameters, is often chosen to be uninformative,\footnote{This does not necessarily mean ``uniform'': there is a method, known as Jeffrey's prior, which allows one to maximize the ``ignorance'' about a parameter once a likelihood is given, through what is basically an argument for reparametrization invariance. For example, an uninformative prior on the mean of a Gaussian is uniform, while an uninformative prior on its standard deviation is log-uniform.} so that we do not introduce bias in the analysis. 
In certain cases it might be warranted to use biased priors: for example, the analysis for GW170817 offers results based on a low-spin and a high-spin prior \cite{abbottGW170817ObservationGravitational2017}: the first is theoretically motivated and might be more meaningful, but the second is still allowed by the data. 

The second ingredient is the likelihood \(\mathbb{P}(s | \vec{\theta})\): this is where signal modelling comes in, since the likelihood needs t include the way the theoretical signal \(h_\theta (t)\) depends on the parameters. 

What is the probability of observing \(s = h_\theta + n\) if we fix \(\theta \)? The theoretical signal \(h_\theta \) can be computed and is thereafter fixed, so this amounts to the probability of observing a certain realization of the noise: under the assumption of Gaussianity, this will read 
%
\begin{align}
\mathbb{P}(h_\theta + n | \vec{\theta}) 
&\propto 
\exp(- \frac{(n|n)}{2}) 
= \exp( - \frac{(s-h_\theta | s-h_\theta )}{2} )
= \exp(
     (s | h_\theta ) 
     - \frac{(s|s)}{2} 
     - \frac{(h_\theta |h_\theta )}{2}
     )  \\
&\propto \exp( (s | h_\theta ) - 
\frac{(h_\theta | h_\theta )}{2})
\,.
\end{align}



\end{document}