\documentclass[main.tex]{subfiles}
\begin{document}

\section{Gravitational Wave theory}

\subsection{Linearized gravity}

The simplest way to discuss gravitational radiation is to consider linearized gravity on a flat Minkowskian background (with no sources, for now). 

We give a brief overview, roughly following the path taken by \textcite{carrollSpacetimeGeometryIntroduction2019}.
[NOT TRUE FIND BETTER SOURCE]
 
This means that we assume that our spacetime admits a reference frame for which the metric is in the form 
%
\begin{align}
g_{\mu \nu } = \eta_{\mu \nu } + h_{\mu \nu }
\,,
\end{align}
%
where the value of the components \(h_{\mu \nu }\) is small enough that we can work to first order in them. 
Any equation in this section includes an implicit ``\(+ \order{h^2}\)''.
We work in this \emph{global inertial frame}. 

In order to study the evolution of the perturbation \(h_{\mu \nu }\) we need to solve the Einstein Field Equations for it to linear order. 
In a vacuum, they can be written as 
%
\begin{align}
G_{\mu \nu } &= R_{\mu \nu} - \frac{1}{2} \eta_{\mu \nu } R  = 0 \\
R_{\mu \nu } &= g^{\alpha \beta } R_{\alpha \mu \beta  \nu }   \\
R &= g^{\mu \nu } R_{\mu \nu }
\,,
\end{align}
%
where \(R_{\alpha \mu \beta \nu } \sim \partial \Gamma + \Gamma \Gamma \) is the Riemann tensor, which is written in terms of derivatives and squares of Christoffel symbols \(\Gamma \): 
%
\begin{align}
\Gamma^{\rho }_{\mu \nu } 
&= \frac{1}{2} g^{\rho \lambda } 
\qty( \partial_{\mu } g_{\nu \lambda } + \partial_{\nu } g_{\lambda \mu } - \partial_{\lambda } g_{\mu \nu }) \\
&= \frac{1}{2} \eta ^{\rho \lambda } 
\qty( \partial_{\mu } h_{\nu \lambda } + \partial_{\nu } h_{\lambda \mu } - \partial_{\lambda } h_{\mu \nu }) 
\marginnote{We use the fact that \(\partial \eta = 0\), and keep only linear order terms.}
\,.
\end{align}

Since the Christoffel symbols are of first order in the perturbation, the term \(\Gamma \Gamma \) in the Riemann tensor is of second order and can be neglected. 
Therefore, the relevant components are 
%
\begin{align}
R_{\mu \nu \rho \sigma } = 2\eta_{\mu \lambda } \Gamma^{\lambda }_{\nu [\sigma , \rho ]}  = h_{\mu [\sigma , \rho] \nu } 
- h_{\nu [\sigma , \rho ] \mu  }
\,,
\end{align}
%
which gives us the following expression for the Ricci tensor: 
%
\begin{align}
R_{\mu \nu } = \frac{1}{2}
\qty( 
    h^{\sigma }{}_{\mu, \sigma \nu  } +
    h^{\sigma }{}_{\nu , \sigma \mu } - 
    h_{, \mu \nu  } -  
    \square h_{\mu \nu } 
)
\,,
\end{align}
%
where \(h = \eta^{\mu \nu } h_{\mu \nu }\) is the trace of the perturbation (computed with respect to the flat metric), while \(\square = \eta^{\mu \nu } \partial_{\mu } \partial_{\nu }\) is the flat space d'Alambertian. 

This, in turn, allows us to write out the Einstein tensor: 
%
\begin{align}
G_{\mu \nu } = \frac{1}{2} \qty(
    h^{\sigma }{}_{\mu, \sigma \nu  } +
    h^{\sigma }{}_{\nu , \sigma \mu } - 
    h_{, \mu \nu  } -  
    \square h_{\mu \nu } -
    \eta_{\mu \nu } h^{\rho \lambda }{}_{, \rho \lambda } + \eta_{\mu \nu } \square h
)
\,.
\end{align}

This can be greatly simplified with two steps:
first, we change variable from the perturbation \(h_{\mu \nu }\) to the \emph{trace-reversed} perturbation \(\overline{h}_{\mu \nu } = h_{\mu \nu } - \eta_{\mu \nu } h / 2\) --- the name comes from the fact that \(\eta^{\mu \nu } \overline{h}_{\mu \nu } = - h\).

This substitution allows us to write the Einstein tensor as 
%
\begin{align}
G_{\mu \nu } = - \frac{1}{2} \square \overline{h}_{\mu \nu } +  \overline{h}_{\alpha (\mu , \nu )}{}^{\alpha } 
- \frac{1}{2} \eta_{\mu \nu } \overline{h}_{\alpha \beta }{}^{, \alpha \beta }
\,.
\end{align}

We will shortly show that it is possible, as a \emph{gauge choice}, to set the divergence of the trace-reversed perturbation to zero: \(\partial^{\mu } \overline{h}_{\mu \nu }= 0\). The gauge imposed by this choice is called the \emph{Hilbert Gauge}, which in terms of the regular perturbation reads 
%
\begin{align}
\partial_{\mu } h^{\mu \nu } - \frac{1}{2} \partial^{\nu } h = 0
\,.
\end{align}

With this choice the Einstein tensor becomes simply 
%
\begin{align}
G_{\mu \nu } = - \frac{1}{2} \square \overline{h}_{\mu \nu }
\,,
\end{align}
%
so the general form of the Einstein equations to linear order will be 
%
\begin{align}
\square \overline{h}_{\mu \nu }= - 16 \pi G T_{\mu \nu }
\,.
\end{align}

\subsubsection{Gauge fixing}

The theory of General Relativity is constructed to be invariant under smooth changes of coordinates: under a map in the form \(x \to x' = x' (x)\) (where \(x'(x)\) is a diffeomorphism\footnote{In the physics parlance this property is known as ``diffeomorphism invariance'', while a mathematician would call the kinds of transformations considered ``isometries''.}).



\end{document}
