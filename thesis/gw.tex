\documentclass[main.tex]{subfiles}
\begin{document}

\section{Gravitational wave theory}

\subsection{Linearized gravity}

The simplest way to discuss gravitational radiation is to consider linearized gravity on a flat Minkowskian background. 

We give a brief overview, roughly following the path taken by \textcite[chapter 1]{maggioreGravitationalWavesVolume2007}.
 
This means that we assume that our spacetime admits a reference frame for which the metric is in the form 
%
\begin{align}
g_{\mu \nu } = \eta_{\mu \nu } + h_{\mu \nu }
\,,
\end{align}
%
where the value of the components \(h_{\mu \nu }\) is small enough that we can work to first order in them. 
Any equation in this section includes an implicit ``\(+ \order{h^2}\)''.
We work in this \emph{global inertial frame}. 

In order to study the evolution of the perturbation \(h_{\mu \nu }\) we need to solve the Einstein Field Equations for it to linear order. 
In a vacuum, they can be written as 
%
\begin{align}
G_{\mu \nu } &= R_{\mu \nu} - \frac{1}{2} \eta_{\mu \nu } R  = 0 \\
R_{\mu \nu } &= g^{\alpha \beta } R_{\alpha \mu \beta  \nu }   \\
R &= g^{\mu \nu } R_{\mu \nu }
\,,
\end{align}
%
where \(R_{\alpha \mu \beta \nu } \sim \partial \Gamma + \Gamma \Gamma \) is the Riemann tensor, which is written in terms of derivatives and squares of Christoffel symbols \(\Gamma \): 
%
\begin{align}
\Gamma^{\rho }_{\mu \nu } 
&= \frac{1}{2} g^{\rho \lambda } 
\qty( \partial_{\mu } g_{\nu \lambda } + \partial_{\nu } g_{\lambda \mu } - \partial_{\lambda } g_{\mu \nu }) \\
&= \frac{1}{2} \eta ^{\rho \lambda } 
\qty( \partial_{\mu } h_{\nu \lambda } + \partial_{\nu } h_{\lambda \mu } - \partial_{\lambda } h_{\mu \nu }) 
\marginnote{We use the fact that \(\partial \eta = 0\), and keep only linear order terms.}
\,.
\end{align}

Since the Christoffel symbols are of first order in the perturbation, the term \(\Gamma \Gamma \) in the Riemann tensor is of second order and can be neglected. 
Therefore, the relevant components are 
%
\begin{align}
R_{\mu \nu \rho \sigma } = 2\eta_{\mu \lambda } \Gamma^{\lambda }_{\nu [\sigma , \rho ]}  = h_{\mu [\sigma , \rho] \nu } 
- h_{\nu [\sigma , \rho ] \mu  }
\,,
\end{align}
%
which gives us the following expression for the Ricci tensor: 
%
\begin{align}
R_{\mu \nu } = \frac{1}{2}
\qty( 
    h^{\sigma }{}_{\mu, \sigma \nu  } +
    h^{\sigma }{}_{\nu , \sigma \mu } - 
    h_{, \mu \nu  } -  
    \square h_{\mu \nu } 
)
\,,
\end{align}
%
where \(h = \eta^{\mu \nu } h_{\mu \nu }\) is the trace of the perturbation (computed with respect to the flat metric), while \(\square = \eta^{\mu \nu } \partial_{\mu } \partial_{\nu }\) is the flat space d'Alambertian. 

This, in turn, allows us to write out the Einstein tensor: 
%
\begin{align}
G_{\mu \nu } = \frac{1}{2} \qty(
    h^{\sigma }{}_{\mu, \sigma \nu  } +
    h^{\sigma }{}_{\nu , \sigma \mu } - 
    h_{, \mu \nu  } -  
    \square h_{\mu \nu } -
    \eta_{\mu \nu } h^{\rho \lambda }{}_{, \rho \lambda } + \eta_{\mu \nu } \square h
)
\,.
\end{align}

This can be greatly simplified with two steps:
first, we change variable from the perturbation \(h_{\mu \nu }\) to the \emph{trace-reversed} perturbation \(\overline{h}_{\mu \nu } = h_{\mu \nu } - \eta_{\mu \nu } h / 2\) --- the name comes from the fact that \(\eta^{\mu \nu } \overline{h}_{\mu \nu } = - h\).

This substitution allows us to write the Einstein tensor as 
%
\begin{align}
G_{\mu \nu } = - \frac{1}{2} \square \overline{h}_{\mu \nu } +  \overline{h}_{\alpha (\mu , \nu )}{}^{\alpha } 
- \frac{1}{2} \eta_{\mu \nu } \overline{h}_{\alpha \beta }{}^{, \alpha \beta }
\,.
\end{align}

We will shortly show that it is possible, as a \emph{gauge choice}, to set the divergence of the trace-reversed perturbation to zero: \(\partial^{\mu } \overline{h}_{\mu \nu }= 0\) (see section \ref{sec:gauge-fixing}). The gauge imposed by this choice is called the \emph{Hilbert Gauge},\footnote{Despite the name, this choice was first suggested by De Sitter to Einstein \cite{kennefickTravelingSpeedThought2007}, who had been previously trying to impose the gauge \(\abs{g}= 1\) \cite[page 688]{1916SPAW.......688E}.} which in terms of the regular perturbation reads 
%
\begin{align} \label{eq:hilbert-gauge}
\partial^{\mu } h_{\mu \nu } - \frac{1}{2} \partial_{\nu } h = 0
\,.
\end{align}

With this choice the Einstein tensor becomes simply 
%
\begin{align}
G_{\mu \nu } = - \frac{1}{2} \square \overline{h}_{\mu \nu }
\,,
\end{align}
%
so the general form of the Einstein equations to linear order will be 
%
%
\boxalign{
\begin{align} \label{eq:wave-equation-einstein}
\square \overline{h}_{\mu \nu }= - 16 \pi G \eval{T_{\mu \nu }}_{\text{linear}}
\,,
\end{align}}
%
where the stress-energy tensor is computed up to first order in the metric perturbation.

\subsubsection{Transformations of the perturbation}

The theory of General Relativity is constructed to be invariant under smooth changes of coordinates: under a map in the form \(x \to x' = x' (x)\) (where \(x'(x)\) is a diffeomorphism\footnote{In the physics parlance this property is known as ``diffeomorphism invariance'', while a mathematician would call the kinds of transformations considered ``isometries'', since we ask that they preserve the metric structure of the manifold.}).

Under such a coordinate transformation the metric transforms like any \((0, 2)\) tensor:
%
\begin{align} \label{eq:metric-transformation-general}
g_{\alpha \beta }' ( x' ) = \pdv{x^{ \mu }}{x^{\prime \alpha }} 
\pdv{x^{ \nu }}{x^{\prime \beta }} 
g_{\mu \nu } ( x)
\,.
\end{align}

A general transformation of this kind may break the condition that \(g = \eta + h\) where \(h\) is small, so in order to preserve our framework we restrict ourselves to a small class of transformations. 

One possibility is to consider infinitesimal transformations in the form 
%
\begin{align}
x^{\mu } \to x^{\prime \mu } = x^{\mu } + \xi^{\mu } (x)
\,,
\end{align}
%
where \(\xi^{\mu }\) is a vector field such that \(\abs{\partial_{\mu } \xi_{\nu }}\) is small --- specifically, the condition to impose is that the first order in \(\partial_{\mu } \xi_{\nu }\) should match the first order in \(h_{\mu \nu }\). 

This condition is all we need in order to write the transformation law for the perturbation: the full equation reads
%
\begin{align}
\eta_{\mu \nu } ' + h_{\mu \nu }' \approx 
\qty(\delta^{\alpha }_{\mu } - \partial^{\alpha } \xi_\mu ) 
\qty(\delta^{\beta  }_{\nu  } - \partial^{\beta  } \xi_\nu  ) 
\qty(\eta_{\alpha \beta } + h_{\mu \nu })
\marginnote{Used \(1 / (1+x) = 1-x + \order{x^2}\).}
\,,
\end{align}
%
so the zeroth order contribution is \(\eta'_{\mu \nu } = \eta_{\mu \nu }\), while the first order one is 
%

%
\boxalign{
\begin{align} \label{eq:perturbation-transformation-infinitesimal}
h^{\prime }_{\mu \nu } = h_{\mu \nu } - 2 \partial_{(\mu } \xi_{\nu )}
\,,
\end{align}}
%
which is our transformation law for the metric perturbation. 

We will also need a transformation law for the trace-reversed perturbation \(\overline{h}_{\mu \nu }\): the trace transforms as \(h' \to h - 2 \partial_{\mu } \xi^{\mu }\), therefore the required law is 
%
\begin{align} \label{eq:perturbation-transformation-infinitesimal-tracereversed}
\overline{h}'_{\mu \nu } = \overline{h}_{\mu \nu } - 2 \partial_{(\mu } \xi_{\nu )} + \eta_{\mu \nu } \partial_{\alpha } \xi^{\alpha }
\,,
\end{align}
%


A second class of transformations is a subset of Lorentz boosts and rotations: substituting \(\pdv*{x^{\mu }}{x^{\prime \nu }} = \Lambda_\nu{}^{\mu }\) into the transformation law \eqref{eq:metric-transformation-general} we find that the flat metric is unchanged, while 
%
\begin{align}
h^{\prime }_{\mu \nu } ( x') = \Lambda_{\mu }{}^{\alpha }
\Lambda_{\nu }{}^{\beta } h_{\alpha \beta }
\,,
\end{align}
%
which may remain in the class of small metric perturbations: this is not guaranteed, but it is true for a certain subset of boosts and for all rotations \cite[]{maggioreGravitationalWavesVolume2007}. 

Finally, the perturbation is invariant under shifts in the form \(x^{\prime \mu }= x^{\mu } + a^{\mu }\). 

\subsubsection{Gauge fixing} \label{sec:gauge-fixing}

Now that we know how the perturbation \(h_{\mu \nu }\) transforms under an infinitesimal transformation, we can use this to impose the condition we want --- specifically, the Hilbert gauge \eqref{eq:hilbert-gauge}.

The way to show that this is possible is to write out the way \(\partial^{\mu } \overline{h}_{\mu \nu }\) transforms for an arbitrary choice of \(\xi \), and to see that with an appropriate choice of \(\xi \) we can always map it to zero. 
The transformation reads 
%
\begin{align}
\partial^{\mu } \overline{h}_{\mu \nu }' &= 
\partial^{\mu } \qty(h_{\mu \nu } - 2 \partial_{(\mu } \xi_{\nu )}) - \frac{1}{2} \partial_{\nu } \qty(\eta^{\alpha \beta } \qty(h_{\alpha \beta } - 2 \partial_{(\alpha } \xi_{\beta )}))  \\
&= \partial^{\mu } \overline{h}_{\mu \nu } - \square \xi_{\nu } - \partial_{\nu } \qty(\partial^{\mu } \xi_{\mu }) + \partial_{\nu } \qty(\partial^{\mu } \xi_{\mu })  \\
&= \partial^{\mu } \overline{h}_{\mu \nu } - \square \xi_{\nu }
\,. 
\end{align}

Therefore, from any starting gauge we must only find a \(\xi_{\nu }\) such that \(\partial^{\mu } \overline{h}_{\mu \nu } = \square \xi_{\nu }\), and we will be in the correct gauge. 
This can always be done, since the D'Alambert equation \(\square f = g\) can always be solved for \(f\) --- if we needed to compute \(\xi_{\nu }\) explicitly (which we typically do not) we could use the Green's function \(G(z)\) for the operator, defined by \(\square G(z) = \delta^{(4)} (z)\). 

While the equation is solvable, the solution is not unique: if we were to define an ``inverse'' of the D'Alambertian it would not be a function but a one-to-many relation. 
Specifically, while keeping fixed the value of \(\square \xi_{\nu }\) we can add any function \(\zeta_\nu \) to \(\xi _\nu  \) as long as \(\square \zeta _\nu = 0 \). 
A trivial example is \(\zeta _\nu = \const\), but other wave-like choices are of more interest. 

These still induce a transformation on \(h_{\mu \nu }\) according to the usual law \eqref{eq:perturbation-transformation-infinitesimal}, and they can be used to further specify the form of the gravitational radiation. 

In terms of \textbf{degrees of freedom}, the full perturbation \(h_{\mu \nu }\) starts with 10 as any symmetric 4D, rank-2 tensor; the four Hilbert gauge conditions \eqref{eq:hilbert-gauge} reduce them to 6, while the four residual gauge conditions will allow us to reduce them to 2. 

Doing so in full generality is not useful for us, let us instead fix the residual gauge in the specific context of a plane wave solution to the wave equation \(\square \overline{h}_{\mu \nu } = 0\).

\subsubsection{Plane gravitational waves}

From an analogy to electromagnetic theory it seems reasonable to work with an ansatz in the form 
%
\begin{align}
\overline{h}_{\mu \nu } = A_{\mu \nu } e^{i k_\alpha x^{\alpha }}
\,,
\end{align}
%
where \(A_{\mu \nu }\) is a constant symmetric tensor.

Imposing the wave equation sets \(k_{\alpha } k^{\alpha } = 0\), and the Hilbert gauge \eqref{eq:hilbert-gauge} condition can be written as \(A_{\mu \nu } k^{\mu } = 0\), where \(A = \eta^{\mu \nu } A_{\mu \nu }\). 

In this framework we can impose the residual gauge condition explicitly: a function which satisfies \(\square \zeta_{\mu } = 0\) is \(\zeta_{\mu } = B_{\mu } \exp(i d_\alpha x^{\alpha })\), where \(d_\alpha \) is a null vector (\(d_\alpha d^{\alpha } = 0\)) while \(B_\mu \) is a generic constant vector.

In these terms, the transformation equation \eqref{eq:perturbation-transformation-infinitesimal-tracereversed} reads 
%
\begin{align}
\overline{h}_{\mu \nu } &\to \overline{h}_{\mu \nu } + \qty(- 2 i B_{(\mu } d_{\nu )}  + i \eta_{\mu \nu } B_{\beta } d^{\beta } ) e^{i d_\alpha x^{\alpha }} \\
A_{\mu \nu } e^{i k_\alpha x^{\alpha }} &\to A_{\mu \nu } e^{i k_\alpha x^{\alpha }} + \qty(-2 i B_{(\mu } d_{\nu )} + i \eta_{\mu \nu } B_{\beta } d^{\beta }) e^{i d_\alpha x^{\alpha }} 
\,.
\end{align}

This tells us that if we set the vector \(d_{\alpha }\) to be equal to \(k_\alpha \) the amplitude \(A_{\mu \nu }\) will transform according to the algebraic system
%
\begin{align}
A_{\mu \nu } \to A_{\mu \nu } - 2 i B_{(\mu } k_{\nu )}
\,,
\end{align}
%
which allows us to impose four conditions on \(A_{\mu \nu }\), one for each of the free components of \(B_{\mu }\). 
It is customary to choose \(A = 0 = A_{0 i }\): these are known together as the \textbf{Transverse-Traceless} gauge. 

The condition \(A = 0\) also means that \(h = 0 = \overline{h}\): thus, form this point onward we can stop distinguishing between \(h_{\mu \nu }\) and  \(\overline{h}_{\mu \nu }\), and for simplicity's sake we write the former. 

If we orient our axes such that \(\vec{k} = k^{i}\) is along the \(\hat{z}\) direction (which means \(k^{\mu } = (k, 0, 0, k)^{\top}\)) the conditions can be written as 
\begin{enumerate}
    \item Hilbert gauge + traceless: \(A_{\mu 0}+ A_{\mu 3} = 0\);
    \item traceless: \(A = - A_{00} + A_{11} + A_{22} + A_{33} = 0 \);
    \item transverse: \(A_{0i} = 0\). 
\end{enumerate}

The Hilbert gauge combined with the transverse conditions show that \(A_{00} = 0\) as well, followed by \(A_{13} = 0\), \(A_{23} = 0\) and \(A_{33} = 0\). 

Finally, the traceless condition imposes \(A_{11} = - A_{22} \).
These conditions tell us that the plane gravitational wave must have the form 
%

%
\boxalign{
\begin{align} \label{eq:TT-gauge-gw}
h_{\mu \nu }(x) = \left[\begin{array}{cccc}
0 & 0 & 0 & 0 \\ 
0 & h_+ & h_\times  & 0 \\ 
0 & h_\times  & - h_+ & 0 \\ 
0 & 0 & 0 & 0
\end{array}\right]
\exp(i k_{\alpha } x^{ \alpha })
\,,
\end{align}}
%
where \(h_+ = A_{11} \) and \(h_\times = A_{12} \). 

\subsubsection{Effect on test masses}

The TT gauge we defined has the rather peculiar characteristic of ``moving with the wave'', so that the position of any observer initially at rest (so, such that \(u^{\mu } (\tau = 0 ) = (1, \vec{0})^{\top}\)) is unchanged: the geodesic equation evaluated at \(\tau = 0\) reads
%
\begin{align}
\eval{\dv[2]{x^{\mu }}{\tau}}_{\tau = 0} + \eval{\Gamma^{\mu }_{\nu \rho } u^{\nu } u^{\rho }}_{\tau = 0} &= 0   \\
\eval{\dv[2]{x^{\mu }}{\tau}}_{\tau = 0} &= - \Gamma^{\mu }_{00} = 0 
\,,
\end{align}
%
since the Christoffel symbols computed \(\Gamma^{\mu }_{00}\) with the TT gauge perturbation \eqref{eq:TT-gauge-gw} all vanish: \(\Gamma^{\mu }_{00} = \eta^{\mu \nu } \qty( 2 g_{\nu 0,0} - g_{00, \nu }) / 2 =  0\).

Does this mean that gravitational waves are merely an artefact, and have no effect on particles? 
No, since while in the TT gauge the \emph{positions} of the points do not change, the \emph{distance} among them does.

In order to understand this effect we can make use of the geodesic equation \cite[section 3.10]{carrollSpacetimeGeometryIntroduction2019}, which states that the acceleration experienced between two geodesics whose four-velocities are both approximately \(u^{\mu }\), separated by a vector \(\xi ^{\mu }\),\footnote{There is a technical note to be made here: in GR the notion of a vector between two points in the manifold is meaningless, however we can get around this problem by considering a one-parameter family of geodesics, and identifying the separation vector between them to be the tangent vector associated to the parameter.} is 
%
\begin{align}
a^{\mu } = R^{\mu }_{\nu \rho \sigma } u^{\nu } u^{\rho } \xi^{\sigma }
\,.
\end{align}

Let us consider the same geodesics as before, whose four-velocity is uniformly \(u^{\mu } \equiv \qty(1, \vec{0})^{\top}\): the acceleration will then be given by the matrix product \(R^{\mu }_{00\sigma } \xi^{\sigma }\). 

These components of the Riemann tensor read \cite[eq.\ 7.106]{carrollSpacetimeGeometryIntroduction2019}: 
%
\begin{align}
R^{\mu }_{00 \sigma } = \frac{1}{2} \ddot{h}^{\mu }_{\sigma }
\,.
\end{align}

Therefore, the temporal component of the acceleration is \(a^{0} \propto \ddot{h}^{0}_{\sigma } = 0 \), while the spatial components read 
%
\begin{align}
a^{i} = \frac{1}{2} \ddot{h}^{i}_{j} \xi^{j}
\,.
\end{align}
%


\subsection{The quadrupole formula}



\end{document}
