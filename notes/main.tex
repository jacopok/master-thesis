\documentclass[11pt]{article}

\usepackage[utf8]{inputenc}
\usepackage{indentfirst}
\usepackage{enumitem}
\usepackage{datetime2}
\usepackage{acronym}
% options (https://tex.stackexchange.com/questions/25520/how-can-i-use-the-latex-acronym-package-and-optionally-create-an-acronym-list-i):
% printonlyused: Only list used acronyms
% withpage: In printonlyused-mode show the page number where each acronym was first used.
% nolist: The option nolist stands for “don’t write the list of acronyms”.
% dua: The option dua stands for “don’t use acronyms”. It leads to a redefinition of \ac and \acp, making the full name appear all the time and suppressing all acronyms but the explicity requested by \acf or \acfp.

\makeatletter
\AtBeginDocument{%
  \renewcommand*{\AC@hyperlink}[2]{%
    \begingroup
      \hypersetup{hidelinks}%
      \hyperlink{#1}{#2}%
    \endgroup
  }%
}
\makeatother

\usepackage{algorithm}
\usepackage{algpseudocode}

\usepackage{textcomp}
\usepackage[T1]{fontenc}
\usepackage{multirow,bigdelim}
\usepackage{float}
\usepackage[caption = false]{subfig}
\usepackage{longtable}
\usepackage{listings}
\usepackage{mathtools}
\DeclareMathOperator{\tr}{Tr}
\usepackage{commath}
\usepackage{slashed}
\usepackage{bbold}
\usepackage{xcolor}
\usepackage{physics}
\newcommand{\lambdabar}{{\mkern0.75mu\mathchar '26\mkern -9.75mu\lambda}}
\usepackage[right=4cm,left=2cm,top=3cm,bottom=3.0cm, marginparwidth=2.7cm, marginparsep=3mm]{geometry}
\usepackage{mdframed}


\usepackage{amsmath}
\usepackage{amsfonts}
\usepackage{amssymb}

\numberwithin{equation}{section}
\usepackage{graphicx}

\usepackage[colorinlistoftodos]{todonotes}
\PassOptionsToPackage{hyphens}{url}
\usepackage[colorlinks=true, allcolors=blue]{hyperref}
\hypersetup{breaklinks=true}

% \urlstyle{same}
\usepackage{siunitx}
\sisetup{separate-uncertainty=true}
% \DeclareSIUnit\parsec{pc}
\usepackage{cancel}
\usepackage{mathrsfs}
\usepackage{marginnote}
\renewcommand*{\marginnotevadjust}{-0.3cm}
\renewcommand*{\marginfont}{\scriptsize}
% \usepackage{fancybox}

\usepackage{footnotebackref}

\usepackage[sc]{mathpazo}
\linespread{1.05}         % Palladio needs more leading (space between lines)
\usepackage[T1]{fontenc}

\newcommand\mybox[1]{%
  \fbox{\begin{minipage}{0.9\textwidth}#1\end{minipage}}}

\newcommand{\const}{\mathrm{const}}

\usepackage[section]{placeins}

\usepackage{tikz}
\usetikzlibrary{shapes,arrows,shadows}


\newcommand{\boxalign}[2][0.986\textwidth]{
  \par\noindent\tikzstyle{mybox} = [draw=black,inner sep=6pt]
  \begin{center}\begin{tikzpicture}
   \node [mybox] (box){%
    \begin{minipage}{#1}{\vspace{-5mm}#2}\end{minipage}
   };
\end{tikzpicture}\end{center}}


\pagestyle{plain}

\author{Jacopo Tissino}

\allowdisplaybreaks
\title{Machine Learning Gravitational Waveforms for Binary Neutron Star mergers: a working summary}
\usepackage[
backend=biber,
style=alphabetic,
sorting=nyt,
urldate=iso8601
]{biblatex}

\addbibresource{Masters_thesis.bib}

\usepackage{subfiles}

\begin{document}

\maketitle
% \tableofcontents

The detection of a Gravitational Wave signal from the \ac{BNS} merger GW170817 \cite[]{abbottGW170817ObservationGravitational2017} opened a new avenue for the direct study of neutron star properties, such as their masses, spins and equations of state.

This is done through Bayesian parameter estimation algorithms, which require the comparison of the experimental signal with several million \cite{lackeyEffectiveonebodyWaveformsBinary2017} simulated waveforms or more.
Speed in the generation of simulated waveforms is, therefore, crucial.

There are three main strategies for the generation of these waveforms. In order of decreasing evaluation time, as well as (typically) decreasing accuracy, they are:
\begin{enumerate}
    \item full \ac{NR} simulations: these are our most accurate possible avenue for the study of \ac{BNS} mergers, but they are so computationally expensive that it is reasonable to simulate at most a few tens of initial condition setups --- these are then used to validate the other methods;
    \item \ac{EOB} simulations: these consist in constructing an effective Hamiltonian for a test particle as a proxy for the two-body system and simulating its evolution --- they can be calibrated by comparison to \ac{NR} simulations;
    \item \ac{PN} waveforms: here we expand the equations of motion of the system in powers of $1/c^2$ and analytically calculate the evolution of the amplitude and phase of the waveform.
\end{enumerate}

The \ac{EOB} and \ac{PN} formalisms can be complemented with a tidal term, which is relevant for \ac{BNS} mergers: neutron stars (as opposed to black holes) can be deformed by each other's gravitational field in a way which depends on the specifics of their equation of state. 

The evaluation times for \ac{EOB} waveforms are dependent on the duration of the waveforms: GW170817 lasted for about two minutes, and the generation of waveforms of this length takes on the order of $100\text{ms}$ with state-of-the-art systems, but as the noise in ground-based interferometers decreases we expect to be able to see a longer and longer section of the waveform, which might bring the evaluation times back above $1\text{s}$. 

Also, the waveforms ought to be generated in the frequency domain, since that form is the one which is used in data analysis;
EOB models, on the other hand, natively generate waveforms in the time domain. 

The work in this thesis is an attempt to make progress in this direction, by developing a machine learning system trained on the (Fourier transforms of the) \ac{EOB} waveforms. 
This builds on the work of \textcite{schmidtMachineLearningGravitational2020}, who developed a similar system to generate Binary Black Hole merger waveforms in the time domain.

The algorithm's basic components are as follows:
\begin{itemize}
    \item a training dataset is generated by considering uniformly distributed tuples of \ac{BNS} parameters in the allowed ranges and calculating the corresponding waveforms with the \ac{EOB} model \texttt{TEOBResumS} \cite[]{nagarTimedomainEffectiveonebodyGravitational2018};
    \item each time-domain waveform is Fourier-transformed and decomposed into phase and amplitude;
    \item each \ac{EOB} waveform is compared to a \ac{PN} one with the same parameters, and a dataset is constructed from the residuals of the \ac{EOB} waveform from the \ac{PN} one;
    \item the dimensionality of the training dataset is reduced through \ac{PCA};
    \item a feed-forward neural network is trained to reconstruct the map between the \ac{BNS} system parameters and the \ac{PCA} components of the waveforms. 
\end{itemize}

The resulting model must be evaluated in terms of both speed and accuracy.
Regarding the former, proper benchmarking will begin at a later stage, but the results obtained by \textcite[]{schmidtMachineLearningGravitational2020} indicate the possibility of obtaining a speedup of one to two orders of magnitude compared to EOB algorithms. 

Regarding the latter, preliminary results indicate the possibility of the model reconstructing \ac{EOB} waveforms with three parameters (the mass ratio \(q\) and the two tidal deformabilities \(\Lambda_1\) and \(\Lambda_2 \)) with a maximal unfaithfulness\footnote{The unfaithfulness between two waveforms \(h_1 (t)\) and \(h_2(t) \) is defined as follows in terms of their Fourier transforms: 
%
\begin{align}
\mathcal{F} [h_1 , h_2 ] = 1 - \max_{t_0, \phi_0 } \frac{(h_1 | h_2 )}{\sqrt{(h_1 | h_2 ) (h_2 | h_2 )}}
\qquad \text{with} \qquad
(h_1 | h_2 ) = 4 \Re \int_{f _{\text{min}}}^{\infty } \frac{h_1^{*} (f) h_2 (f)}{S_n (f)} \dd{f}
\,,
\end{align}
%
where the maximum is evaluated over all possible starting times \(t_0 \) and phases \(\phi_0 \) for the waveforms. The product we define between waveforms is known as the Wiener product, and it also appears when computing the signal-to-noise ratio, as well as in the calculation of the likelihood for parameter estimation. 
} of the order of \(\mathcal{F} \simeq \num{e-3}\).
This is comparable to typical values of the unfaithfulness of EOB waveforms compared to \ac{NR} ones (\(\mathcal{F} \simeq \num{2.5e-3}\) \cite[]{nagarTimedomainEffectiveonebodyGravitational2018}).

This indicates that it is possible for this system to become a part of a gravitational wave data analysis pipeline, enabling a significant speedup at the cost a small, tolerable loss in accuracy.

% Waveforms are needed for both parameter estimation and modelled signal searches
Possible applications of the software developed in this work include modelled signal searches and parameter estimation for both current interferometers (such as LIGO and Virgo) and for future ones, such as Einstein Telescope.

% Possible applications go even beyond the current LIGO-Virgo network: when Einstein Telescope comes online, it will be able to reach much lower frequencies, and therefore detect a waveform for a very long time; the evaluation times of current models do not scale well into this regime, needing several seconds for a single waveform for an initial frequency of \(5 \divisionsymbol \SI{10}{Hz}\) \cite[fig.\ 2]{gambaFastFaithfulFrequencydomain2020}. 

\printbibliography

\acrodef{EOB}{effective one-body}
\acrodef{NR}{numerical relativity}
\acrodef{PN}{post-Newtonian}
\acrodef{PCA}{principal component analysis}
\acrodef{BNS}{binary neutron star}

\end{document}
